\chapter*{III}

Vielleicht war es ein Alb der Perversheit --- oder die sardonische Anziehungskraft dunkler, verborgener Quellen --- die mich meine meine Pläne in diese Richtung ändern ließen. Ich hatte lange im Voraus beschlossen, meine Beobachtungen lediglich auf Architektur zu beschränken und ich beeilte mich sogar um zum Dorfplatz zu gelangen und von dort zeitigen Transport aus dieser verfaulenden Stadt von Tod und Verwesung zu erlangen, doch der Anblick des alten Zadok Allen lenkte mein Denken in neue Bahnen und ließ mich mein Tempo unsicher verlangsamen.

Man hatte mir versichert, dass der alte Mann nur wilde, unzusammenhängende und unglaubliche Legenden andeuten kpnnte und ich war gewarnt worden, dass es nicht sicher war, von den Einheimischen im Gespräch mit ihm gesehen zu werden, doch der Gedanke, zu diesem alten Zeitzeugen des städtischen Verfalls mit Erinnerungen, die zurückgehen zu den frühen Tagen der Schifferei und der Fabriken, war eine Verlockung, der keine Vernunft mich standhalten lassen konnte. Im Grunde sind doch die seltsamsten und verrücktesten Mythen oft lediglich Symbole oder Allegorien, die auf Wahrheit fußen --- und der alte Zadok muss alles mit angesehen, das sich in den letzten neunzig Jahren hier um Innsmouth herum zugetragen hat. Die Neugier flammte in mir auf und überstrahlte Verstand und Vorsicht und in meiner jugendlichen Eitelkeit glaubte ich, ich möge ein Körnchen wahrer Geschichte aus der Welle von verwirrten Übertreibungen gewinnen, die ich mit Rohwhiskey höchstwahrscheinlich aus ihm heraus gewinnen würde.

Mir war bewusst, dass ich ihn nicht hier und jetzt behelligen konnte, denn die Feuerwehrmänner würden dies sicherlich bemerken und etwas einzuwenden haben. Stattdessen, überlegte ich, würde ich mich vorbereiten indem ich etwas schwarz gebrannten Schnaps besorgen würde von einem Ort, wo davon nach den Beschreibungen des Jungen aus dem Laden reichlich vorhanden sein sollte. Danach würde ich ganz beiläufig am Feuerwehrhaus herumlungern und mich zum alten Zadok gesellen, nachdem er zu einem seiner häufigen Streifzüge aufgebrochen war. Der Junge hatte erzählt, dass er sehr rastlos war und selten mehr als eine oder zwei Stunden auf einmal vor der Station saß.

Ein Quart Whiskey war leicht, aber nicht billig hinter einem schäbigen Gemischtwarenladen genau hinter dem Platz in der Eliot Street zu beschaffen. Von dem dreckig aussehenden Kerl, der mich dort erwartete, stierte mich ebenfalls ein Hauch des \glqq Innsmouth-Anblickes\grqq an, doch er war auf seine art recht höflich, wahrscheinlich gewohnt an die Kundschaft geselliger Fremder --- Lastwagenfahrer, Goldkäufer und ähnliche Gestalten --- waren gelegentlich in der Stadt.

Den Platz wieder betretend, sah ich dass das Glück mir hold war, denn --- aus der Paine Street um die Ecke des Gilman House schluefrend --- erblickte ich nicht weniger als die große, dürre, lumpige Gestalt des alten Zadok Allen selbst. Getreu meines Planes, gewann ich seine Aufmerksamkeit indem ich die gerade gekaufte Flasche schwenkte und bald war mir  klar, dass er angefangen hatte, mir sehnsüchtig hinterherzuschlurfen als ich in die Waite Street einbog, auf dem Weg in die verlassenste Gegend, die mir einfiel.

Ich wählte meinen Weg gemäß der Karte, die der Junge aus dem Laden mir angefertigt hatte und nahm ein vollständig aufgegebenes Stück des südlichen Ufers, das ich zuvor besucht hatte, zum Ziel. Die einzigen Leute, die dort zu sehen gewesen waren, waren die Angler auf der weit entfernten Mole und noch ein paar Blocks südlich konnte ich aus deren Sichtweite geraten, hatte ein Paar Sitzplätze in einer stillgelegten Werft gefunden und konnte dann den alten Zadok für Zeit unbeobachtet befragen. Bevor ich die Main Street erreichte konnte ich hinter mir ein schwaches und keuchendes \glqq Hey Mister!\grqq vernehmen und erlaubte es dem alten Mann, mich einzuholen und einige kräftige Schlucke aus der Flasche zu nehmen.

Ich begann, meine Fühler auszustrecken während wir die Water Street entlang gingen und uns in mitten der allgegenwärtigen Trostlosigkeit und wirr geneigten Ruinen nach Süden wandten, doch ich fand das die alte Zunge sich nicht so schnell wie ich erhofft hatte, lösen ließ. Nach einiger Zeit sah ich eine grasbewachsene Öffnung zur See hin zwischen verfallenen Ziegelmauern, jenseits derer sich verunkrauteten Ausdehnungen einer alten Werft aus Lehm und Stein zeigten. Haufen von moosbewachsenen Steinen nahe am Wasser versprachen passable Sitzplätze und der Ort war vor jedem möglichen Blick geschützt durch ein zerstörtes Lagerhaus im Norden. Die Aura von Tod und Verlassenheit war morbide und der Fischgestank fast unerträglich, doch ich hatte beschlossen, mich durch nichts abschrecken zu lassen.

Etwa vier Stunden verblieben mir für Konversation, wenn ich den Acht-Uhr-Bus nach Arkham erwischen wollte und ich begann, dem uralten Säufer mehr Schnaps auszuteilen, während ich meine eigene bescheidene Mahlzeit zu mir nahm. Ich gab acht, bei meinen Ausschenkungen nicht über's Ziel hinauszuschießen, denn ich wollte Zadoks weinselige Geschwätzigkeit nicht in einen Stupor übergehen lassen. Nach einer Stunde schien seine verstohlene Schweigsamkeit allmählich zu verschwinden, doch zu meiner Enttäuschung wich er meinen Fragen über Innsmouth und seine von Schatten heimgesuchte Vergangenheit nach wie vor aus. Er plapperte über aktuelle Themen und zeigte gute Kenntnis der Zeitung und eine starke Tendenz dazu, zu Philosophieren in der Art eines moralisierenden Dörflers.

Zum Ende der zweiten Stunde hin fürchtete ich, mein Quart Whiskey würde nicht ausreichen um Ergebnisse hervorzubringen und fragte mich, ob ich den alten Zadok verlassen und mehr besorgen sollte. Just dann jedoch, eröffnete mir das Glück das, was meine Fragen nicht erbringen konnten und das weitschweifende Gerede des keuchenden Alten nahm eine Wendung, die mich nach vorne lehnen und aufmerksam zuhören ließ. Ich saß mit dem Rücken hin zur fischig stinkenden See, doch er war ihr zugewandt und irgendetwas hatte seinen wandernden Blick auf die flache, weit entfernte Silhouette von Devil Reef gelenkt, das zu der Zeit klar und geradezu faszinierenderweise über die Wellen ragte. Dieser Anblick schien ihm zu missfallen, denn er begann eine Reihe schwache Flüche auszustoßen, die in einem vertraulichen Flüstern und einem wissenden Blick endeten. Er beugte sich zu mir, packte meinen Rockaufschlag und zischte einige Anspielungen, die nicht misszuverstehen waren.
\glqq Das is' wo's alles angefangen hat --- der verfluchte Ort aller Boshaftigkeit wo das tiefe Wasser beginnt. Höllentor --- eine Klippe bis zu deren Grund keine Lotschnur reicht. Der alte Käpt'n Obed hat's getan --- er hat aufn Südseeinseln mehr rausgefund'n als gut für ihn war.

% venter -- Bauch -> Schiffsbauch? to make a venter -> Einen Schiffsbauch voll Ware umschlagen? -> Geschäfte machen?
Alle war'n übel dran damals. Handel brach ein, die Fabriken ham' Aufträge verlor'n, sogar die neuen --- un' die besten von unser'n Leuten verlor'n durch Kaperei im Krieg von 1812 oder mit der Brigg Eliza und der alten Schnau Ranger --- beides Gilman-Schiffe. Obed Marsh, er hatte drei Schiffe zu Wasser --- Brigantine \textit{Columby}, Brigg \textit{Hetty} und die Bark \textit{Sumatry Queen}. Er war der einz'ge Käpt'n der Ostindien- und Pazifik-Handel betrieb, obwohl Esdras Martins Barkentine \textit{Malay Pride} da noch Achtundzwanzig Geschäfte gemacht hat.

Aber 's gab keinen wie Käpt'n Obed, den alten Satansbraten! Hehe! Ich kann ihm nich übel nehmen was er von der Ferne erzählt hat un' wie er die Leute dumm genannt hat, weil sie in die christliche Messe gingen un' ihre Last in Demut und Beschei'nheit getrag'n hab'n! Sagte, die soll'n sich bess're Götter suchen, wie welche von denen auf den Westinnischen sie ha'n, die wür'n guten Fischfang liefern im Austausch für deren Opfer un' wür'n wirklich auf Gebete hör'n!

Matt Eliot, sein erster Maat hat auch viel erzählt, aber er war dageg'n, dass die Leut' irgendwelchen Heidenkram tun! Hat von 'ner Insel östlich von Otaheite erzählt, wo's viele steinerne Ruinen gab', älter als ir'ndwer 's hätte sag'n können, so ähnlich wie die auf Ponape in den Karolinen aber mit Schnitzerei'n von Gesichtern wie die groß'n Statuen auf der Osterinsel. Da war auch 'ne kleine Vulkaninsel in der Nähe mit anderen Schnitzereien --- Ruinen abgeschliffen als wär'n sie einmal unter'm Meer gewesen un' mit Bildern von furchtbar'n Monstern überall.

Also, Matt sagte die Eingebor'nen da hätten all den Fisch gehabt, den sie fangen konnt'n und hatten' Armbänder un' Armreife un' Kopfschmuck aus 'nem seltsamen Gold und voll mit Bildern von Monstern genau wie die in die Ruinen auf der kleinen Insel gehauen waren --- ir'ndwie Fischfrösche oder Froschfische in allen möglichen Position'n als wär'n's Menschen. Keiner hat aus denen rausbekommen wo die all das Zeug herhatten un' all die ander'n Eingeborenen ha'n sich gefragt, wie die soviel Fisch finden konnten, wenn schon die nächste Insel magere Fänge hatte. Matt hat sich das auch gefragt un' auch Käpt'n Obed. Obed hat auch bemerkt, dass viele von den schönen, jungen Leuten auf Nimmerwiederseh'n verschwanden Jahr für Jahr un' dass 's da nich viele Alte gab. Außerdem, dachte er, sah'n manche von den Menschen verdammt merkwürdig aus, sogar für Kanaken.

's brauchte schon Obed um die Wahrheit aus den Ungläub'gen rauszukrieg'n. Ich weiß nich' wie aber er hat Handel für die Gold-Dinger, die die getragen haben angefangen. Hat gefragt wo die herkamen un' ob sie mehr davon beschaffen könn'n un' hat schließlich die ganze Geschichte aus dem alten Häuptling --- Walakea wurde der genannt. Niemand außer Obed hätte dem alten Teufel je geglaubt, aber der Käpt'n konnte die Leute lesen wie Bücher. Hehe! Mir glaubt niemand wenn ich das ihnen erzähl' un' Du wahrscheinlich auch nicht, Junge --- obwohl, wenn ich dich so anseh, Du hast genauso scharfe Augen wie der alte Obed sie hatte.\grqq

Das Geflüster des alten Mannes wurde schwächer und ich fand mich selbst erzitternd ob der fürchterlichen und aufrichtigen Ungeheuerlichkeit seines Tonfalls, obwohl ich wusste, dass seine Erzählung nichts als trunkene Phantasie sein kann.

\glqq Also mein Herr, Obed lernte, dass es da Dinge auf dieser Erde gibt, von denen die meist'n Leute nie gehört ha'n --- un' die sie nicht glauben wür'n wenn sie sie hörten. Anscheinend haben diese Kanaken haufenweise ihre jungen Männer und Frau'n geopfert an irgendso 'ne Art Götterwesen die unter'm Meer lebten und ha'n alle möglichen Gefälligkeiten dafür bekommen. Sie haben die Wesen auf der kleinen Insel mit den selstam'n Ruinen getroffen un' anscheinend sollten die furch'baren Bilder von Froschfischmosntern Bilder von denen sein. Vielleicht war'n das die Viecher von denen all die Meerjungfrau'ngeschichten und so weiter entsprungen sind. Die hatten jede Menge Städte am Meeresgrund un' diese Insel war von dort abgehob'n. Anscheinend lebten ein'ge von den Dingern in den Steinbauten als die Insel sich plötzlich an die Oberfläche erhob. So ha'n die Kanaken Wind davon bekommen, dass sie da unten war'n. Hab'n mit Zeichensprache 'n Handel ausgemacht, sobald sie ihre Angst überkomm'n hatten.

Die Dinger mochten Mensch'nopfer! Hatten sie vor Urzeiten, aber ha'n die Oberwelt aus'n Augen verlor'n nach ein'ger Zeit. Was sie den Opfern angetan ha'n kann ich nich sagen un' ich glaub' Obed war nich allzu scharf drauf zu fragen. Aber die Heiden war'n einverstanden, weil sie 'ne harte Zeit gehabt hatten und generell verzweifelt war'n. Sie haben zweimal im Jahr den Seedingern eine bestimmte Zahl junge Menschen übergeben --- am Vorabend zum Mai und zu Halloween --- ganz regelmäßig. Ha'n außerdem was von dem geschnitzten Krimskrams, den sie hergestellt haben übergeb'n. Im Ge'nzug ha'n die Dinger jede Menge Fisch geliefert --- ha'n die von überall im Meer herbeigetrieben --- un' hier un' da gab's auch so 'n paar von den goldartigen Dingern.

Nun, wie gesagt, ha'n die Eingebor'nen die Dinger auf der kleinen Vulkaninsel getroffen --- sind in Kanus hin mit Opfern un'soweiter un' brachten von da welche von den goldigen Juwel'n wenn's welche gab. Zuerst sind die Viecher nich' auf die Hauptinsel, aber nach 'ner Weile wollten sie dann doch. Scheint als wär' den'n danach gewesen, sich unter die Mensch'n zu mischen un' zusammen Zeremonien an den großen Tagen abhalten --- Maiabend und Hallowe'en. Schau ma', die konnten sowohl im als auch über'm Wasser leben --- nennt man Amphibien, glaub' ich. Die Kanaken haben den' erzählt, dass die anderen Inselvölker sie ausrotten wollen, wenn sie Wind bekommen, dass die hier sind aber die meinten nur, das kümmere sie nich', weil sie die ganze Menschenbrut ausrott'n könnt'n, wenn sie sich die Mühe machen wollten --- das heißt, alle, die nich' bestimmte Zeichen kennen, wie sie einst von den verlor'n'n Großen Alten benutzt wur'n, wer immer die auch war'n. Aber weil sie da keine Lust drauf hatt'n, ha'n sie sich bedeckt gehalt'n wenn ir'ndwer die Insel besucht hat.

Als es darum ging, sich mit den froschig'n Fischen zu paaren, ha'n die Kanaken sich erst mal gescheut aber letztendlich was gelernt, das dem Ganzen ein neues Gesicht gab'. Scheint, dass Menschen eine Art von Verwandtschaft mit diesen Wasserviechern hab'n.--- dass alles lebendige ursprünglich mal aus'm Wasser kam un' nur ein paar Veränderungen braucht um wieder zurück zu könn'n. Die Dinger ha'n den Kanaken erzählt, dass wenn sie ihre Blutlinien misch'n, das Kinder brächte, die erst menschlich aussehen aber später mehr un' mehr wie die Dinger wer'n, bis 's sie letz'en'lich zum Wasser zieht un' sie zum Rest der Viecher da unt'n stoßen. Un' das hier iss der wichtige Teil mein Jung' --- die, die zu Fischviechern wurd'n un' ins Wasser gingen, sterben niemals. Die Dinger sterben nie, außer durch Gewalt.

Nun, mein Herr, scheint als ob zu der Zeit als Obed die Inselbewohner kenn'lernte, die schon voll mit Fischblut von den Tiefwasserdingern waren. Wenn sie alt wurd'n un' sich das zeigte, wur'n sie versteckt bis es sie zum Wasser hin un' vom Land weg zog. Ein'ge war'n stärker davon berührt als and're un' manche hab'n sich nie genug verändert, dass es die zum Wasser zog, aber größtenteils hat sich das genauso gefügt wie die Dinger gesagt hatt'n. Die, die stärker wie die Viecher geboren war'n veränderten sich früher aber die, die fast wie Menschen aussahen, blieben manchmal bis sie über siebzig war'n an Land, obwohl die schon davor öfters auf Probeausflüge runter gingen. Leute, die's zum Wasser gezog'n hatte, kamen generell oft zurück zu Besuch, so dass ein'ge mit ihr'n eig'n fünffach'n Urgroßvätern sprach'n, die das trock'ne Land vor'n paarhundert Jahr'n verlassen hatt'n.

Alle hatten die Idee vom Sterb'n aufgege'n --- außer in Kanu-Krieg'n mit den an'ern Inselbewohnern oder als Opfer an die Meeresgötter tief unt'n oder durch Schlangenbisse, Pest oder gallopierende Krankheit oder ir'ndwas, bevor's sie ins Wasser zog --- sondern erwarteten alle die Veränd'rung, die nach 'ner Weile kein bissl schrecklich mehr war. Sie dachten, was sie draus zog'n war alles wert, was sie dafür aufgege'n hatt'n --- un' ich denk' Obed hat sich das so ähnlich überlegt, als er so über Walakeas Geschichte nachgedacht hat. Aber Walakea gehörte zu den wenig'n die nix von dem Fischblut in sich hatten --- kam aus einer Adelslinie, die sich mit Adelslinien von anderen Inseln verheiratete.

Walakea zeigte Obed eine Menge Riten un' Beschwörungen die mit den Seeviechern zu tun hatten un' hat ihm ein'ge von den Leut'n gezeigt, die sich stark von ihrer Menschenform verändert hatt'n. Irgendwie, obwohl er ihn nie eins von den normalen Dingern aus'm Wasser seh'n ließ. Am Ende gab' er ihm 'n merkwürdiges Dingsbums aus Blei oder so, dass so sagte er, die Fischdinger hervorbringen würde, von je'm Ort im Wasser, wo's ein Nest von denen gäb'. Die Idee war, das zusammen mit den richti'en Gebet'n un' so herabzuwerfen. Walakea erlaubte das, da die Dinger über die ganze Welt verstreut war'n un' jeder, der danach suchte, ein Nest finden konnte un' die herbeirufen könnte, wenn er sie brauchte.

Matt mochte dieses Treiben überhaupt nich' un' wollte, dass Obed sich von der Insel fernhielt, doch der Käp'n war scharf auf Profit un' fand raus, dass er die Golddinger für'n Spottpreis bekommen konnte, so dass es sich lohn'n würde, sich darauf zu spezialisier'n. Die Dinge liefen so für Jahre un' Obed bekam genug von dem Goldzeug, dass er die Raffinerie in Waite's alter, verkommener Walkmühle aufmachen konnte. Er hat's nie gewagt, die Stücke zu verkaufen wie sie war'n, weil die Leute sonst ständig Fragen stell'n würd'n. So oder so kamen seine Matrosen manchmal an so'n Stück un' wur'n das hie un' da los, obwohl sie geschwor'n hatten, Ruhe zu halten. Un' er hat die Frau'n in seiner Familie 'n paar von den Stücken, die für  Menschen passender aussah'n als die meisten, tragen lassen.

Un' dann, ungefähr Achtun'dreißig als ich sieben war --- fand Obed das ganze Inselvolk von einer Reise auf die nächste ausgelöscht. Schien als hätten die and'ren Inselvölker Wind davon bekomm'n was da vor sich ging un' hatten sich der Sache selbst angenomm', Ich glaub' die hatten doch welche von den alten magischen Zeichen von den'n die Seeviecher sagten, dass sie davor als einziges Angst hatt'n. Nich auszudenken was diese Kanaken sich alles zu schnappen trauen, wenn der Meeresboden 'ne Insel hervortreibt mit Ruinen, die älter sin' als die Sintflut. War'n andächtige Typen, denn sie ha'n nix stehen gelassen, weder auf der Hauptinsel noch auf der klein'n Vulkaninsel, bis auf die Ruinen, die zu groß war'n um sie abzureißen. An manchen Stellen lag'n kleine Steine verstreut --- wie Schutzzauber --- mit 'nem Symbol drauf, dass sie heute als Hakenkreuz nennen. War'n wohl die Zeichen der Großen Alten. Die Menschen war'n ausgelöscht un' es gab' keine Spur von irgendwelchem Goldzeugs un' keiner der anderen Kanaken wollte ein Wort über die Sache verlier'n. Ha'n sogar geleugnet, dass da jemals Leute auf der Insel gelebt hatt'n.

\grqq
