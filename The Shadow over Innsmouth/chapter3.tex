\chapter*{III}

Vielleicht war es ein Alb der Perversheit --- oder die sardonische Anziehungskraft dunkler, verborgener Quellen --- die mich meine meine Pläne in diese Richtung ändern ließen. Ich hatte lange im Voraus beschlossen, meine Beobachtungen lediglich auf Architektur zu beschränken und ich beeilte mich sogar um zum Dorfplatz zu gelangen und von dort zeitigen Transport aus dieser verfaulenden Stadt von Tod und Verwesung zu erlangen, doch der Anblick des alten Zadok Allen lenkte mein Denken in neue Bahnen und ließ mich mein Tempo unsicher verlangsamen.

Man hatte mir versichert, dass der alte Mann nur wilde, unzusammenhängende und unglaubliche Legenden andeuten kpnnte und ich war gewarnt worden, dass es nicht sicher war, von den Einheimischen im Gespräch mit ihm gesehen zu werden, doch der Gedanke, zu diesem alten Zeitzeugen des städtischen Verfalls mit Erinnerungen, die zurückgehen zu den frühen Tagen der Schifferei und der Fabriken, war eine Verlockung, der keine Vernunft mich standhalten lassen konnte. Im Grunde sind doch die seltsamsten und verrücktesten Mythen oft lediglich Symbole oder Allegorien, die auf Wahrheit fußen --- und der alte Zadok muss alles mit angesehen, das sich in den letzten neunzig Jahren hier um Innsmouth herum zugetragen hat. Die Neugier flammte in mir auf und überstrahlte Verstand und Vorsicht und in meiner jugendlichen Eitelkeit glaubte ich, ich möge ein Körnchen wahrer Geschichte aus der Welle von verwirrten Übertreibungen gewinnen, die ich mit Rohwhiskey höchstwahrscheinlich aus ihm heraus gewinnen würde.

Mir war bewusst, dass ich ihn nicht hier und jetzt behelligen konnte, denn die Feuerwehrmänner würden dies sicherlich bemerken und etwas einzuwenden haben. Stattdessen, überlegte ich, würde ich mich vorbereiten indem ich etwas schwarz gebrannten Schnaps besorgen würde von einem Ort, wo davon nach den Beschreibungen des Jungen aus dem Laden reichlich vorhanden sein sollte. Danach würde ich ganz beiläufig am Feuerwehrhaus herumlungern und mich zum alten Zadok gesellen, nachdem er zu einem seiner häufigen Streifzüge aufgebrochen war. Der Junge hatte erzählt, dass er sehr rastlos war und selten mehr als eine oder zwei Stunden auf einmal vor der Station saß.

Ein Quart Whiskey war leicht, aber nicht billig hinter einem schäbigen Gemischtwarenladen genau hinter dem Platz in der Eliot Street zu beschaffen. Von dem dreckig aussehenden Kerl, der mich dort erwartete, stierte mich ebenfalls ein Hauch des \glqq Innsmouth-Anblickes\grqq an, doch er war auf seine art recht höflich, wahrscheinlich gewohnt an die Kundschaft geselliger Fremder --- Lastwagenfahrer, Goldkäufer und ähnliche Gestalten --- waren gelegentlich in der Stadt.

Den Platz wieder betretend, sah ich dass das Glück mir hold war, denn --- aus der Paine Street um die Ecke des Gilman House schluefrend --- erblickte ich nicht weniger als die große, dürre, lumpige Gestalt des alten Zadok Allen selbst. Getreu meines Planes, gewann ich seine Aufmerksamkeit indem ich die gerade gekaufte Flasche schwenkte und bald war mir  klar, dass er angefangen hatte, mir sehnsüchtig hinterherzuschlurfen als ich in die Waite Street einbog, auf dem Weg in die verlassenste Gegend, die mir einfiel.

Ich wählte meinen Weg gemäß der Karte, die der Junge aus dem Laden mir angefertigt hatte und nahm ein vollständig aufgegebenes Stück des südlichen Ufers, das ich zuvor besucht hatte, zum Ziel. Die einzigen Leute, die dort zu sehen gewesen waren, waren die Angler auf der weit entfernten Mole und noch ein paar Blocks südlich konnte ich aus deren Sichtweite geraten, hatte ein Paar Sitzplätze in einer stillgelegten Werft gefunden und konnte dann den alten Zadok für Zeit unbeobachtet befragen. Bevor ich die Main Street erreichte konnte ich hinter mir ein schwaches und keuchendes \glqq Hey Mister!\grqq vernehmen und erlaubte es dem alten Mann, mich einzuholen und einige kräftige Schlucke aus der Flasche zu nehmen.

Ich begann, meine Fühler auszustrecken während wir die Water Street entlang gingen und uns in mitten der allgegenwärtigen Trostlosigkeit und wirr geneigten Ruinen nach Süden wandten, doch ich fand das die alte Zunge sich nicht so schnell wie ich erhofft hatte, lösen ließ. Nach einiger Zeit sah ich eine grasbewachsene Öffnung zur See hin zwischen verfallenen Ziegelmauern, jenseits derer sich verunkrauteten Ausdehnungen einer alten Werft aus Lehm und Stein zeigten. Haufen von moosbewachsenen Steinen nahe am Wasser versprachen passable Sitzplätze und der Ort war vor jedem möglichen Blick geschützt durch ein zerstörtes Lagerhaus im Norden. Die Aura von Tod und Verlassenheit war morbide und der Fischgestank fast unerträglich, doch ich hatte beschlossen, mich durch nichts abschrecken zu lassen.

Etwa vier Stunden verblieben mir für Konversation, wenn ich den Acht-Uhr-Bus nach Arkham erwischen wollte und ich begann, dem uralten Säufer mehr Schnaps auszuteilen, während ich meine eigene bescheidene Mahlzeit zu mir nahm. Ich gab acht, bei meinen Ausschenkungen nicht über's Ziel hinauszuschießen, denn ich wollte Zadoks weinselige Geschwätzigkeit nicht in einen Stupor übergehen lassen. Nach einer Stunde schien seine verstohlene Schweigsamkeit allmählich zu verschwinden, doch zu meiner Enttäuschung wich er meinen Fragen über Innsmouth und seine von Schatten heimgesuchte Vergangenheit nach wie vor aus. Er plapperte über aktuelle Themen und zeigte gute Kenntnis der Zeitung und eine starke Tendenz dazu, zu Philosophieren in der Art eines moralisierenden Dörflers.

Zum Ende der zweiten Stunde hin fürchtete ich, mein Quart Whiskey würde nicht ausreichen um Ergebnisse hervorzubringen und fragte mich, ob ich den alten Zadok verlassen und mehr besorgen sollte. Just dann jedoch, eröffnete mir das Glück das, was meine Fragen nicht erbringen konnten und das weitschweifende Gerede des keuchenden Alten nahm eine Wendung, die mich nach vorne lehnen und aufmerksam zuhören ließ. Ich saß mit dem Rücken hin zur fischig stinkenden See, doch er war ihr zugewandt und irgendetwas hatte seinen wandernden Blick auf die flache, weit entfernte Silhouette von Devil Reef gelenkt, das zu der Zeit klar und geradezu faszinierenderweise über die Wellen ragte. Dieser Anblick schien ihm zu missfallen, denn er begann eine Reihe schwache Flüche auszustoßen, die in einem vertraulichen Flüstern und einem wissenden Blick endeten. Er beugte sich zu mir, packte meinen Rockaufschlag und zischte einige Anspielungen, die nicht misszuverstehen waren.
\glqq Das is' wo's alles angefangen hat --- der verfluchte Ort aller Boshaftigkeit wo das tiefe Wasser beginnt. Höllentor --- eine Klippe bis zu deren Grund keine Lotschnur reicht. Der alte Käpt'n Obed hat's getan --- er hat aufn Südseeinseln mehr rausgefund'n als gut für ihn war.

% venter -- Bauch -> Schiffsbauch? to make a venter -> Einen Schiffsbauch voll Ware umschlagen? -> Geschäfte machen?
Alle war'n übel dran damals. Handel brach ein, die Fabriken ham' Aufträge verlor'n, sogar die neuen --- un' die besten von unser'n Leuten verlor'n durch Kaperei im Krieg von 1812 oder mit der Eliza und der alten Ranger-Schnau --- beides Gilman-Schiffe. Obed Marsh, er hatte drei Schiffe zu Wasser --- Brigantine \textit{Columby}, Brigg \textit{Hetty} und die Bark \textit{Sumatry Queen}. Er war der einz'ge Käpt'n der Ostindien- und Pazifik-Handel betrieb, obwohl Esdras Martins Barkentine \textit{Malay Pride} da noch bis Achtundzwanzig Geschäfte gemacht hat.
\grqq
