\chapter*{II}

Um kurz vor Zehn am nächsten Morgen stand ich mit einer kleinen Reisetasche vor Hammond's Drug Store am alten Marktplatz und wartete auf den Innsmouth-Bus. Als die Stunde seiner Ankunft nahte, bemerkte ich ein allgemeines Abwandern der anwesenden Müßiggänger zu anderen Orten, die Straße hoch oder zum Ideal Lunch auf der anderen Seite des Platzes. Offensichtlich hatte der Schalterbeamte nicht übertrieben, was die Abneigung, die die hiesigen Leute gegenüber Innsmouth und seinen Bewohnern hegten, anging. Nach wenigen Augenblicken ratterte ein kleiner Autobus von extremer Klapprigkeit und dreckig grauer Farbe die State Street herunter, wendete und hielt am Bordstein neben mir. Mir war sofort bewusst, dass dies der Richtige war. Ein Eindruck, den das halb lesbare Schild in der Windschutzscheibe --- \glqq Arkham-Innsmouth-Newb'port\grqq --- bald bestätigte.

Da waren nur drei Fahrgäste --- dunkle, ungepflegte Männer von mürrischem Antlitz und einigermaßen jugendlicher Gestalt --- und als das Fahrzeug stoppte, schlurften sie unbeholfen heraus und begannen, die State Street auf stille, fast verstohlene Art entlangzugehen. Der Fahrer stieg ebenfalls aus und ich beobachtete ihn, wie er in die Drogerie ging um einen Einkauf zu tätigen. Dies, dachte ich, muss jener Joe Sargent sein, den der Fahrkartenverkäufer erwähnt hatte. Und noch bevor ich irgendwelche Details bemerkte, überkam mich eine Welle der spontanen Abneigung, die ich weder kontrollieren, noch erklären konnte. Es kam mir plötzlich sehr selbstverständlich vor, dass die Leute hier nicht in einem Bus fahren wollten, der im Besitz dieses Mannes war und von ihm gefahren wurde oder den Lebensraum eines solchen Mannes und Seinesgleichen öfter als nötig zu besuchen.

Als der Fahrer wieder aus dem Geschäft trat, schaute ich ihn genauer an und versuchte, die Ursache meines üblen Eindrucks auszumachen. Er war ein dünner Mann mit hängenden Schultern, knapp unter sechs Fuß groß, gekleidet in lumpiger blauer Zivilkleidung und einer ausgefransten Golfmütze. Sein Alter betrug vielleicht fünfunddreißig Jahre, doch die seltsamen, tiefen Falten in den Seiten seines Halses ließen ihn  älter aussehen, wenn man nicht sein träges, ausdrucksloses Gesicht betrachtete. Er hatte einen schmalen Kopf, hervortretende, feuchte, blaue Augen, die niemals zu blinken schienen, eine flache Nase, fliehende Stirn und Kinn und unentwickelte Ohren. Seine langen, dicken Lippen und grobporigen, grauen Wangen schienen fast bartlos, mit Ausnahme einiger spärlicher, gelber Haare, die in ungleichmäßigen Büscheln wucherten und sich kräuselten und an einigen Stellen schien die Hautoberfläche eigenartig auffällig, als ob sie sich durch eine Hautkrankheit schäle. Seine Hände waren groß, schwer von Venen durchzogen und hatte eine ungewöhnliche, grau-blaue Färbung. Die Finger waren auffallend kurz in Proportion zum Rest und schienen die Neigung zu haben, sich fest in die riesige Handfläche zu ballen. Während er zum Bus ging, bemerkte ich seine eigentümliche, watschelnde Gangart und sah, dass seine Füße übermäßig groß waren. Je mehr ich sie studierte, desto mehr fragte ich mich, wie er wohl passende Schuhe finden konnte.

Eine gewisse, ihm anhaftende Schmierigkeit verstärkte meinen Missfallen. Er arbeitete dem Anschein nach an den Docks für die Fischer --- oder hing dort herum --- und trug sehr viel von ihrem charakteristischen Gestank mit sich. Welches fremdartige Blut genau sich in ihm wieder fand, konnte ich beim besten Willen nicht erraten. Seine Eigentümlichkeiten sahen sicherlich nicht asiatisch, polynesisch, morgenländisch oder negrid aus, doch sah ich ein, warum die Leute ihn als fremdartig empfanden. Ich selbst hätte eher biologische Degeneration angenommen, als fremde Herkunft.

Ich bedauerte es, als ich sah, dass keine anderen Fahrgäste im Bus sein würden. Irgendwie mochte ich den Gedanken nicht, alleine mit diesem Busfahrer zu fahren. Doch als die Abfahrtszeit sich näherte, überkam ich meine Bedenken und folgte dem Mann an Bord, reichte ihm einen Dollarschein und murmelte das Wort \glqq Innsmouth\grqq. Er schaute mich kurz neugierig an, während er mir wortlos vierzig Cent Wechselgeld herausgab. Ich nahm einen Platz weit hinter ihm, jedoch auf der selben Seite des Busses ein, da ich während der Reise die Küste verfolgen wollte.

Schließlich startete das heruntergekommene Vehikel mit einem Zucken und ratterte geräuschvoll entlang der alten Backsteinbauten der State Street in einer Wolke von Auspuffgasen. Die Leute auf den Bürgersteigen flüchtig betrachtend, glaubte ich in ihnen einen eigentümlichen Wunsch zu entdecken, Blickkontakt mit dem Bus --- oder zumindest den Eindruck, so auszusehen als starre man ihn an --- zu vermeiden. Dann bogen wir links auf die High Street ab, wo es schneller voranging, an imposanten alten Villen aus der Frühzeit der Republik und noch älteren Farmhäusern aus der Kolonialzeit vorbei, bevor sie in eine lange, eintönige Strecke offenen Küstenlandes hervortrat.

Der Tag war warm und sonnig, doch die Landschaft aus Sand, Riedgras und Zwerggesträuch wurde immer trostloser, je weiter wir kamen. Aus dem Fenster konnte ich das blaue Wasser und die sandigen Umrisse von Plum Island sehen und bald kamen wir dem Strand sehr nahe, als unsere schmale Strecke von der Hauptstraße nach Rowley und Ipswich abdrehte. Es waren keine Häuser zu sehen und ich konnte am Zustand der Fahrbahn erkennen, dass hier ziemlich wenig Verkehr entlang kam. Die kleinen, verwitterten Telefonmasten trugen nur zwei Leitungen. Hier und da überfuhren wir krude, hölzerne Brücken über Priele, die sich weit ins Land hinein wanden und zur generellen Isolation der Region beitrugen.

Gelegentlich bemerkte ich tote Baumstümpfe und bröckelnde Fundamente in den Dünen und erinnerte mich an die alte Überlieferung in einer der alten Geschichten, die ich gelesen hatte, dass dies einst eine fruchtbare und dicht besiedelte Landschaft war. Die Veränderung kam, sagte man, mit der Epidemie von 1846 in Innsmouth und wurde vom einfachen Volk in dunkle Verbindung mit geheimen, bösen Mächten gebracht. In Wirklichkeit, war sie durch die törichte Abholzung der Küstenwälder entstanden, die die Erde ihres besten Schutzes beraubte und den Weg für Sandwehen freimachte.

Schließlich verloren wir Plum Island aus den Augen und sahen die riesige Fläche des Atlantik zu unserer Linken. Unser schmaler Parcours begann, steil anzusteigen und ich spürte ein eigenartiges Gefühl der Unruhe, als ich den einsamen Kamm voraus erblickte, wo die zerfurchte Straße den Himmel traf. Es war als würde der Bus seinen Anstieg immer weiter fortsetzen, den festen Grund komplett hinter sich lassen und mit den verborgenen Geheimnissen der Höhen des geheimnisvollen Himmels verschmelzen. Der Geruch der See brachte unheilvolle Andeutungen mit sich und der verkrümmte, steife Rücken und schmale Kopf des Fahrers wurden mir mehr und mehr verhasst. Während ich ihn anschaute, bemerkte ich, dass die Rückseite seines Kopfes fast so haarlos war, wie sein Gesicht mit nur ein paar wuchernden gelben Strähnen auf der grauen, schuppigen Haut.

Dann erreichten wir den Gipfel und betrachteten das sich jenseits davon erstreckende Tal, wo der Manuxet in die See mündet, nördlich der langen Reihe an Klippen die sich in Kingsport Head vollenden und dann nach Cape Ann ausscheren. Am fernen, diesigen Horizont konnte ich die luftigen Umrisse des Head ausmachen, gekrönt von jenem seltsamen, uralten Haus über das so viele Legenden erzählt werden, doch für den Moment wurde meine ganze Aufmerksamkeit von einem näheren Rundblick, direkt unter mir eingefangen. Ich befand mich, so wurde mir klar, im Angesicht des von sagenumwobenen Innsmouth.

Es war eine ausgedehnte, dicht gebaute Stadt, doch mit einem unheilvollen Mangel an sichtbarem Leben. Aus dem Durcheinander von Schornsteinspitzen war kaum eine Rauchfahne zu sehen und drei hohe Türme ragten kahl und unbemalt gegen den den seewärtigen Horizont. Einer von ihnen bröckelte an der Spitze und darin, sowie in einem weiteren prangten nur schwarze, gähnende Löcher wo Zifferblätter sich hätten befinden müssen. Der riesige Haufen absackender Mansardendächer und spitzer Giebel ließ mit abstoßender Klarheit die Vorstellung wurmstichigen Verfalls deutlich werden und als wir entlang der nun abfallenden Straße näher kamen, konnte ich sehen, dass viele Dächer komplett eingestürzt waren. Es gab auch einige große, viereckige Georgianische Häuser mit Walmdächern, Kuppeln und von Geländern umrandeten Witwenstegen. Diese waren größtenteils ein stückweit weg vom Wasser und ein oder zwei schienen in halbwegs intaktem Zustand. Ich erblickte die verrostete, überwucherte, sich landeinwärts erstreckende Linie der stillgelegten Eisenbahnschienen, samt schief stehenden Telegrafenmasten ohne Leitungen und die halbverdeckten Wege der alten Fahrbahn nach Rowley und Ipswich.

In Ufernähe war der Verfall am schlimmsten, obwohl ich in seiner Mitte den weißen Glockenturm eines recht gut erhaltenen Backsteinbauwerks, das wie eine kleine Fabrik aussah, erspähen konnte. Der Hafen, schon lange versandet, war umschlossen von einem uralten steinernen Wellenbrecher auf dem ich die winzigen Formen einiger sitzender Fischer ausmachen konnte und an dessen Ende die Fundamente von etwas lagen, das aussah wie ein längst verfallener Leuchtturm. Eine sandige Landzunge hatte sich innerhalb der Barriere geformt und auf ihr saßen einige verfallene Häuschen, einige festgemachte Dories und verstreute Hummerkäfige. Die einzige tiefe Stelle schien da zu sein wo der Fluss an dem Gebäude mit dem Glockenturm vorbei floss und sich dann nach Süden wandte um am Ende der Mole in den Ozean zu münden.

Hier und da ragten die Ruinen von Werften aus der Küste um in unbestimmter Fäulnis zu enden. Die verfallensten von ihnen schienen am weitesten im Süden zu stehen. Und weit draußen im Meer konnte ich trotz der Flut eine lange, schwarze Linie erblicken, die sich kaum aus dem Wasser erhob, jedoch eine Andeutung sonderbarer, ruhender Bösartigkeit in sich barg. Ich wusste, dies musste Devil Reef sein. Während ich es betrachtete, schien sich ein subtiler, eigenartiger Ruf zu meiner düsteren Ablehnung hinzu zu mischen und eigentümlicherweise empfand ich diesen Unterton verstörender als den Haupteindruck.

Wir trafen niemanden auf der Straße, aber kamen bald an verlassenen Farmen in unterschiedlichem Verfallszustand vorbei. Dann bemerkte ich ein paar unbewohnte Häuser, deren zerbrochenen Fenster mit Lumpen gestopft und deren Höfe übersät waren mit Muschelschalen und toten Fischen. Ein- oder zweimal sah ich teilnahmslos wirkende Leute in den kargen Gärten arbeiten oder im Sand des nach Fisch stinkenden Strandes nach Muscheln graben und Gruppen von dreckigen, affengesichtigen Kindern auf den unkrautüberwucherten Schwellen spielen. Auf irgendeine Art wirkten diese Menschen beunruhigender als die trostlosen Gebäude, denn fast alle von ihnen hatten gewisse Eigenheiten in Antlitz und Bewegungen, die mir instinktiv missfielen ohne sie definieren oder verstehen zu können. Eine Sekunde lang dachte ich, dieser typische Körperbau deutete auf irgendein Bild zurück, das ich einmal gesehen hatte, vielleicht in einem Buch, in einer Situation von außergewöhnlicher Abscheu oder Wehmut, doch diese Pseudoerinnerung verflüchtigte sich schnell wieder.

Als der Bus die tieferliegende Ebene erreichte, begann ich den stetigen Ton eines Wasserfalls durch die unnatürliche Stille wahrzunehmen. Die schiefen, blanken Häuser standen jetzt dichter, säumten beide Seiten der Straße und zeigten eine mehr innerstädtische Richtung als diejenigen, die wir nun hinter uns ließen. Das Panorama voraus hatte sich zu einer Straßenszene zusammengezogen und an manchen Stellen konnte ich erkennen, wo Kopfsteinpflaster und Flecken von geziegelten Bürgersteinen einmal existiert hatten. All diese Häuser waren anscheinend verlassen und es gab gelegentliche Lücken wo zusammenfallende Kamine und Kellerwände an Bauwerke erinnerten, die eingestürzt waren. Dies alles wurde durchdrungen vom ekelerregendsten erdenklichen Fischgestank.

Bald tauchten Querstraßen und Kreuzungen auf; jene zur Linken führten küstenwärts in Gefilde unbefestigten Elends und Verfalls, während die zur Rechten den Blick auf vergangene Größe freigaben. Bis jetzt hatte ich in der Stadt keine Menschen gesehen, doch nun kamen mir vereinzelte Zeichen von Bewohnung entgegen --- Fenster mit Vorhängen hier und da und gelegentlich ein zerbeultes Automobil am Straßenrand. Pflaster und Gehsteige waren besser umrissen und obwohl die meisten Häuser sehr alt waren --- alte Holz- und Ziegelbauten aus dem frühen 19. Jahrhundert --- wurden sie offensichtlich in bewohnbarem Zustand gehalten. Als Hobbyantiquar verlor ich fast meinen Ekel vor dem Geruch und mein Gefühl von Bedrohung und Abstoßung im Angesicht dieses reichhaltigen, unveränderten, überlebenden Stückes Vergangenheit.

Doch ich erreichte mein Ziel nicht ohne einen starken Eindruck von geradezu bewegend widerwärtiger Art. Der Bus war an einer Art offenem Versammlungsplatz oder Kreisel angelangt, mit Kirchen auf zwei Seiten und den Resten einer runden Grünfläche in der Mitte und ich schaute zu einer großen, säulengesäumten Halle an der Abfahrt rechter Hand voraus. Die einstmals weiße Farbe des Gebäudes war nun grau und begann abzublättern und das schwarz-goldene Schild unter dem Giebel war so verblasst, dass ich nur schwer die Worte \glqq Esoterischer
Orden des Dagon\grqq  ausmachen konnte. Dies war also der ehemalige Freimaurertempel, der nun zu einem niederen Kult übergegangen war. Während ich mich anstrengte, diese Inschrift zu entziffern, wurde meine Beobachtung durch die rauen Töne einer gesprungenen Glocke auf der anderen Straßenseite gestört. und ich drehte mich schnell um aus dem Fenster auf meiner Seite des Busses zu  schauen.

Der Lärm kam aus dem plump wirkenden Turm einer steinernen Kirche aus deutlich späterer Zeit als die meisten Häuser, gebaut in einem unbeholfenem gotischen Stil und mit einem unverhältnismäßig hohen Keller mit verriegelten Fensterläden. Obwohl die Zeiger der Uhr auf der Seite, die ich betrachtete fehlten, wusste ich, dass diese heiseren Glockenschläge elf Uhr ansagten. Dann wurden plötzlich alle Gedanken an die Zeit ausgelöscht durch ein hereinbrechendes eindringliches Bild von unerklärlichem Schrecken, das mich ergriffen hatte, bevor ich begriffen hatte, worum es sich handelte. Das Tor des Kirchenkellers stand offen und zeigte darin ein schwarzes Rechteck. Und als ich hereinschaute, kreuzte ein bestimmtes Objekt dieses dunkle Rechteck --- oder schien es zu kreuzen --- und brannte in meinen Verstand eine augenblickliche, alptraumhafte Vorstellung, die noch unerträglicher wirkte, weil eine Auseinandersetzung damit nicht eine einzige alptraumhafte Eigenschaft darin hervorbrachte.

Es war ein lebendes Wesen --- das erste außer dem Fahrer, das ich gesehen hatte, seit wir in den inneren Bereich der Stadt vorgedrungen waren --- und wäre ich in einer beständigeren Stimmung gewesen, hätte ich darin nichts von Schrecken gefunden. Es war, wie mir einen Moment später einleuchtete, ganz klar der Pastor, gehüllt in sonderbare Gewänder, die zweifelsohne eingeführt worden waren, seit der Orden des Dagon die Rituale der hiesigen Kirchen verändert hatte. Das, was wahrscheinlich meinen ersten, unbewussten Blick eingefangen und den Hauch von bizarrem Entsetzen ausgelöst hatte, war die hohe Tiara, die er trug; ein fast exaktes Duplikat derjenigen, die Miss Tilton mir am Abend vorher gezeigt hatte. Dies hatte durch meine Vorstellungskraft ungenannte, sinistre Eigenschaften an das unbestimmte Gesicht und die zugehörige, schlurfende, in Roben gekleidete Gestalt verliehen. Es gab, wie ich bald beschloss, keinen Grund warum ich diese schauderhafte Berührung böser Pseudo-Erinnerungen erlebt haben sollte. War es nicht ganz selbstverständlich, dass ein lokaler, geheimnisvoller Kult in seiner Kleiderordnung eine einzigartige Sorte von Kopfbedeckung aufnehmen würde, die der Gemeinde auf irgendeine seltsame Weise bekannt war --- etwa als Teil eines Schatzes?

Langsam wurden, dünn versprengt, abstoßend aussehende jüngere Leute auf den Bürgersteigen sichtbar --- einzelne Individuen oder stille Gruppen von zwei bis drei. Die unteren Stockwerke der bröckelnden Häuser beherbergten manchmal kleine Läden mit schmuddeligen Schildern und ich bemerkte gelegentlich geparkte Lastwagen an denen wir vorbeiratterten. Das Geräusch von Wasserfällen wurde immer deutlicher und bald sah ich eine recht tiefe Schlucht vor uns, die von einer breiten, von eisernen Geländern gesäumte Brücke überspannt wurde, auf deren anderer Seite sich ein Platz ausbreitete. Als wir über die Brücke schepperten, schaute ich nach beiden Seiten heraus und erspähte einige Fabrikgebäude an der Kante der grasbedeckten Klippe und auf halbem Weg darunter. Weit unter mir floss das Wasser reichlich und ich konnte zwei Gruppen von kraftvollen Wasserfällen zu meiner rechten und mindestens einen weiteren stromabwärts zu meiner Rechten sehen. An diesem Punkt war das Getöse ohrenbetäubend. Dann rollten wir auf den großen, halbkreisförmigen Platz jensetis des Flusses und hielten an der rechten Seite vor einem hohen, kuppelgekrönten Gebäude mit Überresten von gelber Farbe und einem halblesbaren Schild, das es als \textit{Gilman House} auswies.

Ich war froh, den Bus zu verlassen und schritt sogleich voran, um meine Reisetasche in der schäbigen Hotellobby zu hinterlegen. Dort war nur eine Person in Sicht -- ein älterer Mann ohne das, was ich das \glqq Innsmouth-Aussehen\grqq  getauft hatte --- und ich entschied mich, ihm keine der Fragen zu stellen, die mich plagten, als ich mich an die merkwürdigen Dinge erinnerte, die in diesem Hotel bemerkt worden waren. Stattdessen schlenderte ich nach draußen auf den Platz, von dem aus der Bus schon weitergefahren war und beobachtete den Anblick minuziös und abwägend.

Auf der einen Seite des gepflasterten, offenen Platzes sah man die gerade Linie des Flusses, auf der anderen einen Halbkreis aus schräg überdachten Ziegelbauten aus der Zeit um 1800, aus dem mehrere Straßen nach Südosten, Süden und Südwesten abgingen. Es gab erbärmlich wenige und kleine Lampen --- allesamt schwache Glühlampen --- und ich war froh, dass meine Pläne eine Abreise vor Einbruch der Dunkelheit vorsahen, obwohl ich wusste, dass der Mond hell scheinen würde. Die Gebäude waren allesamt in angemessenem Zustand und fassten vielleicht ein Dutzend geöffnete Läden unter denen sich ein Lebensmittelgeschäft der Kette First National, ein tristes Restaurant, eine Drogerie, zwei Fischgroßhändler, sowie am östlichsten Punkt des Platzes ein Büro der einzigen Industrie dieser Stadt --- der Marsh Refining Company. Man konnte vielleicht zehn Menschen sehen und vier oder fünf Autos und Laster standen verstreut herum. Ich konnte mir selbst denken, dass dies das Gemeindezentrum von Innsmouth war. Zum Osten konte ich flüchtige Blicke der blauen Umrisse des Hafens erhaschen, gegen die sich die verfallenden Überreste von drei einstmals schönen Georgianischen Kirchtürmen erhoben. Und zur Küste hin auf der anderen Seite des Flusses sah ich den weißen Glockenturm, der das was ich für die Marsh-Raffinerie hielt überragte.

Aus irgendeinem Grund entschied ich mich, meine ersten Erhebungen im Lebensmittelgeschäft anzustellen, dessen Personal wahrscheinlich nicht aus Innsmouth stammen würde. Ich fand einen einzelnen Jungen von vielleicht siebzehn Jahren in Dienst und war froh über seine Heiterkeit und Freundlichkeit, die mir muntere Information versprachen. Er schien außergewöhnlich begierig, sich zu unterhalten und ich fand schnell heraus, dass er den Ort, seinen Fischgeruch und seine verstohlenen Leute nicht mochte. Ein Wortwechsel mit einem Außenstehenden war eine Befreiung für ihn. Er kam aus Arkham, lebte bei einer Familie aus Ipswich und ging zurück nach heim wann immer er eine Minute dienstfrei bekam. Seine Familie mochte nicht, dass er in Innsmouth arbeiten musste aber die Kette hatte ihn hierher versetzt und er wollte seinen Job nicht verlieren.

Es gab, so sagte er, keine öffentliche Bibliothek oder Handelskammer in Innsmouth, doch ich könne mich wahrscheinlich zurechtfinden. Die Straße von der ich gekommen war, war die Federal. Westlich davon waren die feinen, alten Wohnstraßen --- Broad, Washington, Lafayette und Adams --- und östlich waren die küstennahen Elendsviertel. Es war in diesen Armensiedlungen --- und entlang der Hauptstraße --- wo ich die alten Georgianischen Kirchen finden würde, doch sie wären alle schon lange verlassen. Ich täte gut daran, mich in diesen Gegenden nicht zu auffällig zu verhalten --- insbesondere nördlich des Flusses --- da dort die Leute düster und feindselig seien. Einige Fremde wären sogar verschwunden.

Manche Orte waren quasi verbotenes Gebiet, wie er unter beträchtlichem Aufwand gelernt hatte. Man sollte zum Beispiel nicht zu lange in der Nähe der Marsh-Raffinerie oder einer der noch genutzten Kirchen, oder um die säulengesäumte Ordenshalle des Dagon am New Church Green. Diese Kirchen waren seltsam --- allesamt von ihren Konfessionen anderswo heftigst abgeleugnet und sie nutzten offensichtlich die wunderlichsten Zeremonielle und geistlichen Gewänder. Ihre Glaubensbekenntnisse waren heterodox und mysteriös und umfassten Andeutungen über gewisse wundersame Transformationen, die zu einer Form von körperlicher Unsterblichkeit auf dieser Erde führen sollen. Der Pastor des Jungen, Dr. Wallace von der Bischöflichen Methodistenkurche von Asbury in Arkham, hatte ihn dringend ermahnt, keiner Kirche in Innsmouth beizutreten.

Und was die Einwohner von Innsmouth anging, so wusste der Junge kaum, was er mit ihnen anfangen sollte. Sie waren so verstohlen und so selten anzutreffen wie Tiere, die in Bauten lebten und man mochte sich kaum vorstellen, wie sie sich die Zeit vertrieben, außer durch die sporadische Fischerei. Vielleicht --- anhand der Menge von Schwarzgebranntem, das sie konsumierten zu urteilen --- lagen sie tagsüber die meisten Stunden, vom Alkoholrausch benommen darnieder. Sie schienen in ihrem Missmut in einer Art Gemeinschaft und gemeinsamer Auffassung vereint, die Welt zu verschmähen, als gäbe es andere, dieser vorzuziehende Sphären des Daseins. Ihr Aussehen --- besonders die starrenden Augen, die nie zu blinzeln schienen --- war sicherlich schockierend genug und ihre Stimmen klangen abscheulich. Es war entsetzlich, sie in ihren Kirchen des nachts singen zu hören, besonders zu ihren Hauptfestivitäten oder Erneuerungszeremonien, die zweimal im Jahr, am 30. April und am 31. Oktober stattfanden.

Sie waren dem Wasser zugetan und schwammen häufig, sowohl im Fluss als auch im Hafenbecken. Häufig fanden Schwimmwettkämpfe raus zum Devil Reef statt und jedermann schien fähig, sich an diesem anstrengenden Sport zu beteiligen. Wenn man es sich recht überlegt, so waren in der Öffentlichkeit generell nur eher junge Leute zu sehen und von diesen schienen die ältesten am entstelltesten auszusehen. Ausnahmen waren meist Leute ohne jede Spur von Anomalie, wie der alte Angestellte im Hotel. Man mochte sich fragen, was aus der Masse an älteren Leuten wurde und ob das \glqq Innsmouth-Aussehen\grqq  nicht ein seltsames und heimtückisches Krankheitssymptom sei, das sein Opfer mehr und mehr ergriff, während die Jahre vorbeigingen.

Natürlich könnte nur ein sehr seltenes Leiden solche enormen und drastischen anatomischen Veränderungen in einem einzelnen ausgewachsenen Individuum hervorrufen --- Veränderungen, die Einfluss auf den Knochenbau bis zur Grundform des Schädels nahmen --- jedoch war nicht einmal dieser Gesichtspunkt rätselhafter oder beispielloser als die sichtbaren Züge der Krankheit im Ganzen. Es sei schwer, so bemerkte der Junge, zu diesem Thema irgendwelche wirklichen Schlüsse zu ziehen, da man die Einwohner nie persönlich kennenlerne, ganz gleich wie lang man in Innsmouth leben möge.

Der Jüngling war sich sicher, dass Exemplare, noch schlimmer als die schlimmsten sichtbaren, in einigen der  Häuser eingeschlossen gehalten wurden. Manchmal hörten Leute die seltsamsten Geräusche. Die wackeligen  Bruchbuden nördlich des Flusses waren angeblich durch geheime Tunnel verbunden und boten somit ein wahres Labyrinth ungesehener Abartigkeiten. Welch fremdes Blut --- falls überhaupt welches --- in diesen Wesen floss, war unmöglich zu sagen. Sie hielten manchmal diverse besonders widerwärtige Gestalten außer Sichtweite, wenn Regierungsbeamte und andere aus der Außenwelt in die Stadt kamen.

Es wäre sinnlos, so mein Informant, die Einwohner irgendetwas über den Ort zu fragen. Der einzige, der reden mochte, war ein sehr alter, doch normal aussehender Mann, der im Armenhaus am nördlichen Rand der Stadt wohnte und seine Zeit herumstreunend oder vor dem Feuerwehrhaus lungernd herumbrachte. Dieser ergraute Kerl, Zadok Allen, war sechsundneunzig Jahre alt, etwas wirr im Kopf und außerdem der städtische Trunkenbold. Er war ein eigenartiges, verstohlenes Geschöpf und schaute ständig über seine Schulter als ob er in Furcht vor etwas lebe. Wenn er nüchtern war, konnte er überhaupt nicht dazu überredet werden, mit Fremden zu sprechen. Er war jedoch nicht im Stande, seinem liebsten Gift zu widerstehen und sobald er betrunken war, würde er die erstaunlichsten Fragmente geflüsterter Erinnerungen von sich geben.

Insgesamt jedoch, könnten kaum sinnvolle Einzelheiten von ihm gewonnen werden, da seine Geschichten allesamt irrsinnige, unvollständige Andeutungen unmöglichen Wunders und Entsetzens die keinen anderen Ursprung als seine eigene gestörte Fantasie haben können. Niemand glaubte ihm jemals, doch die Einwohner mochten nicht, wenn er trank und mit Fremden sprach und es war nicht immer sicher, gesehen zu werden wie man ihn befragte. Er war es wahrscheinlich von dem sich einige der wildesten bekannten Gerüchte und Irrtümer ableiteten.

Mehrere nicht einheimische Anwohner hatten von Zeit zu Zeit von grässlichen Eindrücken berichtet, doch mit Zadoks Geschichten und den missgebildeten Bewohnern war es kein Wunder, dass solche Halluzinationen üblich geworden waren. Niemand der Fremden blieb jemals bis spät in die Nacht draußen, da es den weit verbreiteten Eindruck machte, dass es nicht klug sei, das zu tun. Außerdem waren die Straßen unheimlich dunkel.

Was die Geschäfte anging --- die Fülle an Fisch war sicherlich verblüffend, doch die Einheimischen nutzten dies weniger und weniger. Zudem fielen die Preise und die Konkurrenz erstarkte. Natürlich lag das eigentliche Geschäft der Stadt in der Raffinerie, deren Handelsbüro am Dorfplatz sich nur ein paar Türen östlich von unserem Standort befand. Der Alte Marsh war nie zu sehen, fuhr jedoch manchmal in einem geschlossenen, verhangenen Wagen zur Arbeit.

Es kursierten alle möglichen Gerüchte darüber, wie Marsh mittlerweile aussähe. Er war einmal ein großer Dandy gewesen und man sagte, er trüge noch immer einen feinen Gehrock aus der Zeit Edwards VII, eigentümlich an gewisse Missbildungen angepasst. Seine Söhne hatten einst das Büro am Platz geführt, doch in letzter Zeit hielten sie sich meistens verborgen und überließen die Hauptlast der Geschäfte der jüngeren Generation. Die Söhne und ihre Schwestern sahen mittlerweile auch sehr seltsam aus, besonders die älteren und man erzählte sich, dass ihre Gesundheit versage.

Eine der Marsh-Töchter war eine  abstoßende, einem Reptil ähnlich sehende Frau, die ein Übermaß an merkwürdigem Schmuck, der ganz klar zur gleichen exotischen Tradition wie die fremdartige Tiara gehörte. Mein Informant hatte dies oft bemerkt und hatte gehört, wie man darüber sprach, dass dies aus einer Art geheimem Schatz stamme, entweder von Piraten oder Dämonen angelegt. Die Geistlichen --- oder Priester, oder wie auch immer sie hier nun genannt wurden --- trugen solche Schmuckstücke ebenfalls als Kopfschmuck, doch bekam man sie nur selten zu Gesicht. Mehr Exemplare hatte der Junge nicht gesehen, obwohl die Existenz von vielen weiteren in Innsmouth gerüchtet wurde.

Die Marshes, zusammen mit den anderen drei vornehmen Familien der Stadt --- den Waites, den Gilmans und den Eliots --- lebten allesamt sehr zurückgezogen. Sie wohnten in immensen Häusern entlang der Washington Street und mehrere davon beherbergten vermeintlich einige lebende Verwandtschaft, deren Anblick ihnen Öffentlichkeit verbat und deren Tod bereits gemeldet und dokumentiert war.

Der junge Mann malte für mich eine grobe, aber ausreichende und gewissenhaft gezeichnete Karte der auffälligsten Merkmale der Stadt und warnte mich, dass viele der Straßenschilder fehlten. Nach einer kurzen Analyse war ich mir sicher, dass sie von großem Nutzen sein würde, bedankte mich überschwänglich und steckte sie in die Tasche. Da mir die Schäbigkeit des einzigen Restaurants, das ich gesehen hatte nicht zusagte, kaufte ich einen anständigen Vorrat an Käsegebäck und Ingwerplätzchen, die mir später als Mittagessen dienen sollten. Mein Programm, so entschied ich, sah es vor, mich entlang der Hauptstraßen zu hangeln und mit jeglichen nicht einheimischen Menschen zu sprechen, die ich treffen möge um danach den Acht-Uhr-Bus nach Arkham zu nehmen. Die Stadt, soviel war sichtbar, formte ein signifikantes und überspitztes Beispiel gemeindlichen Verfalls, doch ich, der ich kein Sozialwissenschaftler war, würde meine ernsthaften Beobachtungen auf das Feld der Architektur beschränken.

So begann ich meine systematische, doch halb verdutzte Tour durch Innsmouth's enge, vom Schatten verdunkelte Straßen. Ich überquerte die Brücke und wendete mich dem Getöse der unteren Wasserfälle zu. So kam ich der Marsh-Raffinerie nahe, die seltsam frei von Industrielärm schien. Sie stand auf der Klippe zum Fluss nahe einer Brücke und einem offenen Zusammenfluss von Straßen, die ich als das frühere Stadtzentrum betrachtete, das  sich nach der Revolution zum jetzigen Marktplatz verlagert hatte.

Als ich die Schlucht auf der Main Street Bridge wieder überquerte, traf ich auf einen äußerst verlassenen Bereich, bei dem mich ein ungewisser Schauder überkam. Ein Durcheinander zusammengefallener Giebeldächer formten eine zerklüftete und fantastische Silhouette, über der sich der makabere, enthauptete Turm einer uralten Kirche erhob. Einige Häuser entlang der Hauptstraße waren bewohnt, doch die meisten waren fest verbarrikadiert. Unbefestigte Seitenstraßen hinunter erblickte ich die schwarzen, klaffenden Fenster verlassener Hütten, von denen viele aufgrund ihres sinkenden Fundaments in gefährlichen und unglaublichen Winkeln lehnten.
Diese Fenster starrten so gespenstisch, dass es mich Überwindung kostete, mich ostwärts dem Ufer zuzuwenden. Sicher schwillt der Schrecken eines verlassenen Hauses eher in Form einer geometrischen, denn einer arithmetischen Folge mit der Zahl der Häuser, die sich mehren um eine Stadt gänzlicher Verwüstung zu formen. Der Anblick solch endloser Alleen von fischäugiger Leere und Tod und der Gedanke an unendlich viele verbundene, düstere Räume, aufgegeben an Spinnweben, Erinnerungen und den Eroberer Wurm, bringen rudimentäre Furcht und Abscheu hervor, das gefestigtste Weltbild zu zerstreuen vermag.

Fish Street war so verlassen wie die Hauptstraße, unterschied sich aber durch zahlreiche geziegelte und steinerne Lagerhäuser, die sich noch immer in ausgezeichnetem Zustand befanden. Water Street glich ihr beinahe gänzlich, bis auf die großen Lücken zum Ufer hin, wo einst Werften gestanden hatten. Ich begegnete keinem lebenden Wesen bis auf die versprengten Angler weit draußen an der Mole und ich hörte nichts bis auf das Plätschern der Wellen im Hafen und dem Tosen der Wasserfälle des Manuxet. Diese Stadt begann, mir mehr und mehr auf die Nerven zu gehen und ich schaute heimlich hinter mich als ich meinen Weg zurück über die wackelige Water Street Bridge nahm. Die Fish Street Bridge lag laut der Zeichnung in Trümmern.

Nördlich des Flusses gab es Spuren verwahrlosten Lebens --- Aktive Fischfabriken, rauchende Schornsteine und geflickte Dächer hier und da, gelegentliche Geräusche aus ungewissen Quellen und sporadische, watschelnde Silhouetten in den düsteren Straßen und unbefestigten Gassen --- doch schien mir dies noch beklemmender als die Verlassenheit im Süden. Zum einen waren die Menschen hässlicher und abartiger als jene näher zum Stadtzentrum hin, so dass ich mehrmals übel an etwas absolut fantastisches erinnert wurde, das ich nicht zuordnen konnte. Zweifellos trat die fremdartige Verformung der Einwohner von Innsmouth hier stärker zu Tage als weiter landeinwärts --- es sei denn, das \glqq Innsmouth-Aussehen\grqq war eine Krankheit und keine Blutlinie, in welchem Fall dieser Stadtteil die weiter fortgeschrittenen Fälle beherbergte.

Ein Detail, das mich beunruhigte war die Verteilung der wenigen, schwachen Geräusche, die ich vernahm. Sie sollten normalerweise aus den sichtbar bewohnten Häusern kommen, doch waren sie in Wirklichkeit oft am lautesten aus den am stärksten verbarrikadierten Fassaden zu hören. Man konnte Knarzen, Trippeln und zweifelhafte, heisere Laute wahrnehmen und ich musste unbehaglich an die geheimen Tunnel denken, die der Junge vom Lebensmittelladen angedeutet hatte. Plötzlich ertappte ich mich dabei, wie ich mich fragte, wie wohl die Stimmen dieser Bewohner klingen mögen. Ich hatte bisher noch kein gesprochenes Wort in diesem Viertel vernommen und war unerklärlich bestrebt dies auch so zu belassen.

Ich hastete aus dem abscheulichen Elendsviertel am Hafen und hielt nur lang genug an um mir zwei schöne, aber verfallene alte Kirchen in der Main und der Church Street anzuschauen. Mein nächstes logisches Ziel wäre New Church Green gewesen, doch irgendwie konnte ich es nicht fertigbringen, noch einmal an der Kirche vorbeizugehen, in deren Keller ich die unfassbar furchteinflößende Gestalt des mit dem seltsamen Diadem gekrönten Pastor erspäht hatte. Außerdem hatte mir der Junge aus dem Laden erzählt, dass die Kirchen sowie die Ordenshalle des Dagon keine Viertel waren, in denen Fremde sich sehen lassen sollten.

Dementsprechend hielt ich mich nördlich, entlang der Main zur Martin, dann landeinwärts über die Federal Street, weit nördlich vom Green und betrat dann das verfallene Patrizierviertel der nördlichen Broad, Washington, Lafayette und Adams Street. Obwohl diese imposanten, alten Alleen schlecht befestigt und verwahrlost waren, war der Geist ihrer von Ulmen beschatteten Würde noch nicht ganz verflogen. Villa an Villa verlangte nach meinem Blick, die meisten baufällig und verbarrikadiert, in mitten vernachlässigter Grundstücke, doch ein bis zwei in jeder Straße zeigten Zeichen von Bewohnung. In der Washington Street stand eine Reihe von vier oder fünf in ausgezeichnetem Zustand mit fein gepflegten Rasen und Gärten. Die üppigste von Ihnen --- mit von weitläufigen Terrassen gesäumten Parterregärten, die sich bis zur Lafayette zurück erstreckten --- hielt ich für das Heim des alten Marsh, dem leidgeplagten Raffineriebesitzer.

In all diesen Straßen war kein Lebewesen zu sehen und ich wunderte mich ob der Abwesenheit von Hunden und Katzen in Innsmouth. Außerdem verblüfften und beängstigten mich die vielen, sogar in den besterhaltenen Villen, komplett verschlossenen Fenster im dritten Stock und im Dachgeschoss. Heimlichkeit und Verschwiegenheit schienen in dieser totgeschwiegenen Stadt voll von Fremdartigkeit und Tod allgegenwärtig und ich konnte dem Gefühl nicht entkommen, von allen Seiten aus dem Hinterhalt von durchtrieben starrenden, sich niemals schließenden Augen beobachtet zu werden.

Ich erzitterte als die gesprungene Glocke in einem Turm zu meiner linken Drei schlug. Nur zu gut erinnerte ich mich an die plumpe Kirche aus der diese Töne drangen. Ich folgte der Washington Street zum Fluss hin und sah mich nun einem ehemaligen Industrie- und Handelsviertel gegenüber, bemerkte die Ruine einer Fabrik voraus und sah weitere, mit den Spuren eines alten Bahnhofs und einer überdachten Eisenbahnbrücke jenseits davon, oberhalb der Schlucht zu meiner Rechten.

Die unsichere Brücke vor mir war mit einem Warnschild versehen, doch ich nahm das Risiko auf mich und überquerte sie um wieder zum südlichen Ufer zurückzukehren, wo die Spuren des Lebens wieder zum Vorschein kamen. Verstohlene, watschelnde Kreaturen starrten hintergründig in meine Richtung und normalere Gesichter betrachteten mich kalt und neugierig. Innsmouth wurde rasch unerträglich und ich bog in die Paine Street in Richtung des Platzes in der Hoffnung, dass mich irgendein Fahrzeug vor der noch in weiter Ferne liegenden Abfahrtszeit jenes finsteren Busses nach Arkham bringen könnte.

Da sah ich das baufällige Spritzenhaus zu meiner Linken und bemerkte den rotgesichtigen, rauschebärtigen, triefäugigen alten Mann in unscheinbaren Lumpen, der davor auf einer Bank saß und mit zwei zerzausten, aber nicht unnormal aussehenden Feuerwehrleuten sprach. Dies musste natürlich Zadok Allen sein, der halb verrückte, trunksüchtige Neunzigjährige, dessen Märchen vom alten Innsmouth und seinem Schatten so scheußlich und unglaublich waren.
