\chapter*{II}

Um kurz vor Zehn am nächsten Morgen stand ich mit einer kleinen Reisetasche vor Hammond's Drug Store am alten Marktplatz, auf den Innsmout-Bus wartend. Als sich die Stunde seiner Ankunft näherte, bemerkte ich ein allgemeines Abwandern der Müßiggänger zu anderen Stellen die Straße hoch oder zum Ideal Lunch auf der anderen Seite des Platzes. Offensichtlich hatte der Schalterbeamte nicht übertrieben, was die Abneigung, die die hiesigen Leute gegenüber Innsmouth und seinen Bewohnern hegten, anging. Nach wenigen Augenblicken ratterte ein kleiner Autobus von extremer Klapprigkeit und dreckig grauer Farbe die Stae Street herunter, wendete und hielt am Bordstein neben mir. Mir war sofort bewusst, dass dies der richtige war. Ein Eindruck, den das halbunlesbare Schild in der Windschutzscheibe --- \glqq Arkham-Innsmouth-Newb'port\grqq --- bald bestätigte.

Da waren nur drei Fahrgäste --- dunkle, ungepflegte Männer von mürrischer Miene und einigermaßen jugendlicher Gestalt --- und als das Fahrzeug stoppte, schlurften sie unbeholfen heraus und begannen, die State Street auf stille, fast verstohlene Art entlangzugehen. Der Fahrer stieg ebenfalls aus und ich beobachtete ihn, wie er in die Drogerie ging um einen Einkauf zu tätigen. Dies, dachte ich, muss jener Joe Sargent sein, den der Fahrkartenverkäufer erwähnt hatte. Und noch bevor ich irgendwelche Details bemerkte, überkam mich eine Welle der spontanen Abneigung, die ich weder kontrollieren, noch erklären konnte. Es kam mir plötzlich sehr selbstverständlich vor, dass die Leute hier nicht in einem Bus fahren wollten, der im Besitz dieses Mannes war und von ihm gefahren wurde oder den Lebensraum eines solchen Mannes und Seinesgleichen öfter als nötig zu besuchen.

Als der Fahrer wieder aus dem Geschäft trat, schaute ich ihn genauer an und versuchte, die Ursache meines bösartigen Eindrucks auszumachen. Er war ein dünner Mann mit hängenden Schultern, knapp unter sechs Fuß groß, gekleidet in lumpiger blauer Zivilkleidung und eine ausgefranste Golfmütze. Sein Alter betrug vielleicht fünfunddreißig, doch die seltsamen, tiefen Falten in den Seiten seines Halses ließen ihn  älter aussehen, wenn man nicht sein träges, ausdrucksloses Gesicht betrachtete. Er hatte einen schmalen Kopf, hervortretende, wässrige, blaue Augen, die niemals zu blinken schienen, eine flache Nase, fliehende Stirn und Kinn und unentwickelte Ohren. Seine langen, dicken Lippen und grobporigen, grauen Wangen schienen fast bartlos, mit Ausnahme einiger spärlicher, gelber Haare, die in ungleichmäßigen Flecken wucherten und sich kräuselten und an einigen Stellen schien die Hautoberfläche eigenartig auffällig aus, als ob sie sich durch eine Hautkrankheit schäle. Seine Hände waren groß, schwer von Venen durchzogen und hatte eine ungewöhnliche, grau-blaue Färbung. Die Finger waren auffallend kurz in Proportion zum Rest und schienen die Neigung zu haben, sich fest in die riesige Handfläche zu ballen. Während er zum Bus ging, bemerkte ich seine eigentümliche, watschelnde Gangart und sah, dass seine Füße übermäßig groß waren. Je mehr ich sie studierte, desto mehr fragte ich mich, wie er wohl passende Schuhe finden konnte.

Eine gewisse, ihm anhaftende Schmierigkeit verstärkte meineen Missfallen. Er arbeitete dem Anschein nach an den Docks für die Fischer --- oder hing dort herum --- und trug sehr viel von ihrem charakteristischen Gestank mit sich. Welches fremdartige Blut genau sich in ihm wieder fand, konnte ich beim besten Willen nicht erraten. Seine Seltsamkeiten sahen sicherlich nicht asiatisch, polynesisch, morgenländisch oder negrid aus, doch sah ich ein, warum die Leute ihn als fremdartig empfanden. Ich selbst hätte eher an biologische Degeneration gedacht, als an fremde Herkunft.

Ich bedauerte es, als ich sah, dass keine anderen Fahrgäste im Bus sein würden. Irgendwie mochte ich den Gedanken nicht, alleine mit diesem Busfahrer zu fahren. Doch als die Abfahrtszeit sich näherte, überkam ich meine Bedenken und folgte dem Man an Bord, reichte ihm einen Dollarschein und murmelte das Wort \glqq Innsmouth\grqq. Er schaute mich kurz neugierig an, während er mir wortlos vierzig Cent Wechselgeld herausgab. Ich nahm einen Platz weit hinter ihm, jedoch auf der selben Seite des Busses ein, da ich während der Reise die Küste verfolgen wollte.

Schließlich starte das heruntergekommene Vehikel mit einem Zucken und ratterte geräuschvoll entlang der alten Backsteinbauten der State Street in einer Wolke aus Auspuffgasen. Die Leute auf den Bürgersteigen flüchtig betrachtend, glaubte ich, einen eigentümlichen Wunsch, Blickkontakt mit dem Bus zu vermeiden --- oder zumindest zu vermeiden, so auszusehen als starre man ihn an --- zu entdecken. Dann bogen wir links auf die High Street ab, wo es schneller voranging, an imposanten alten Villen aus der Frühzeit der Republik und noch älteren Farmhäusern aus der Kolonialzeit vorbei, bevor sie in eine lange, eintönige Strecke offenen Küstenlandes hervortrat.

Der Tag war warm und sonnig, doch die Landschaft aus Sand, Riedgras und Zwerggesträuch wurde immer trostloser, je weiter wir kamen. Aus dem Fenster konnte das blaue Wasser und die sandigen Umrisse von Plum Island sehen und bald kamen wir dem Strand sehr nahe als unsere schmale Strecke von der Hauptstraße nach Rowley und Ipswich abdrehte. Es waren keine Häuser zu sehen und ich konnte am Zustand der Fahrbahn erkennen, dass hier ziemlich wenig Verkehr entlang kam. Die kleinen, verwitterten Telefonmasten trugen nur zwei Leitungen. Hier und da überfuhren wir krude, hölzerne Brücken über Priele, die sich weit ins Land hinein wanden und zur generellen Isolation der Region beitrugen.

Gelegentlich bemerkte ich tote Baumstümpfe und bröckelnde Fundamente in den Dünen und erinnerte mich an die alte Überlieferung in einer der alten Geschichten, die ich gelesen hatte, dass dies einst eine fruchtbare und dicht besiedelte Landschaft war. Die Veränderung kam, sagte man, mit der Epidemie von 1846 in Innsmouth und wurde vom einfachen Volk in dunkle Verbindung mit geheimen, bösen Mächten gebracht. In Wirklichkeit, war sie durch die törichte Abholzung der Küstenwälder entstanden, die die Erde ihres besten Schutzes beraubte und den Weg für Sandwehen freimachte.

Schließlich verloren wir Plum Island aus den Augen und sahen die riesige Fläche des Atlantik zu unserer Linken. Unser schmaler Parcours begann, steil anzusteigen und ich spürte ein eigenartiges Gefühl der Unruhe, als ich den einsamen Kamm voraus erblickte, wo die zerfurchte Straße den Himmel traf. Es war als würde der Bus seinen Anstieg immer weiter fortsetzen, den soliden Grund komplett unter sich lassen und mit den verborgenen Geheimnissen der Höhen des geheimnisvollen Himmels verschmelzen. Der Geruch der See brachte unheilvolle Andeutungen mit sich und der verkrümmte, steife Rücken und schmale Kopf des Fahrers wurden mir mehr und mehr verhasst. Während ich ihn anschaute, bemerkte ich, dass die Rückseite seines Kopfes fast so haarlos war, wie sein Gesicht mit nur ein paar wuchernden gelben Strähnen auf der grauen, schuppigen Haut.

Dann erreichten den Gipfel und betrachteten das sich jenseits davon erstreckende Tal, wo der Manuxet in die See mündet, nürdlich der langen Reihe an Klippen sich in Kingsport Head vollenden und dann nach Cape Ann ausscheren. Am fernen, diesigen Horizont konnte ich die luftigen Umrisse des Head ausmachen, gekrönt von jenem seltsamen, uralten Haus über das so viele Legenden erzählt werden, doch für den Moment wurde meine ganze Aufmerksamkeit von einem näheren Rundblick, direkt unter mir eingefangen. Ich befand mich, so wurde mir klar, im Angesicht des von Gerüchten umwobenen Innsmouth.

Es war eine ausgedehnte, dicht gebaute Stadt, doch mit einem unheilvollen Mangel an sichtbarem Leben. Aus dem Durcheinander von Schornsteinspitzen war kaum eine Rauchfahne zu sehen und drei hohe Türme ragten kahl und unbemalt gegen den den seewärtigen Horizont. Einer von ihnen bröckelte an der Spitze und darin und  in einem weiteren prangten nur schwarze, gähnende Löcher wo Zifferblätter sich hätten befinden müssen. Der riesige Haufen absackender Mansardendächer und spitzer Giebel ließ mit abstoßender Klarheit die Vorstellung wurmstichigen Verfalls deutlich werden und als wir entlang der nun abfallenden Straße näher kamen, konnte ich sehen, dass viele Dächer komplett eingestürzt waren. Es gab auch einige große, viereckige Georgianische Häuser mit Walmdächern, Kuppeln und von Geländern umrandeten Witwenstegen. Diese waren größtenteils ein stückweit weg vom Wasser und ein oder zwei schienen in angemessen intaktem Zustand. Sich landeinwärts erstreckend erblickte ich die verrostete, überwucherte Linie der stillgelegten Eisenbahnschienen, samt schief stehenden Telegrafenmasten ohne Leitungen und die halbverdeckten Wege der alten Fahrbahn nach Rowley und Ipswich.

Der Verfall war nahe des Ufers am schlimmsten, obwohl ich in seiner Mitte den weißen Glockenturm eines recht gut erhaltenen Backsteinbauwerks, das wie eine kleine Fabrik aussah, erspähen konnte. Der Hafen, schon lange versandet, waru umschlossen von einem uralten steinernen Wellenbrecher auf dem ich die winzigen Formen einiger sitzender Fischer ausmachen konnte und an dessen Ende die Fundamente von etwas lagen, das aussah wie ein längst verfallener Leuchtturm. Eine sandige Landzunge hatte sich innerhalb der Barriere geformt und auf ihr saßen einige verfallene Häuschen, einige festgemachte Dories und verstreute Hummerkäfige. Die einzige tiefe Stelle schien da zu sein wo der Fluss an dem Gebäude mit dem Glockenturm vorbei floss und sich dann nach Süden wandte um am Ende der Mole in den Ozean zu münden.

Hier und da ragten die Ruinen von Werften aus der Küste um in unbestimmter Morschheit zu enden. Die verfallendsten von ihnen schienen am weitesten im Süden zu stehen. Und weit draußen im Meer konnte ich trotz der Flut eine lange, schwarze Linie erblicken, die sich kaum aus dem Wasser erhob, jedoch eine Andeutung sonderbarer, ruhender Bösartigkeit in sich barg. Ich wusste, dies musste Devil Reef sein. Während ich es betrachtete, schien sich ein subtiler, eigenartiger Ruf zu der düsteren Ablehnung hinzu zu mischen und eigentümlicherweise empfand ich diesen Unterton verstörender als den Haupteindruck.

Wir trafen niemanden auf der Straße, aber kamen bald an verlassenen Farmen in unterschiedlichem Verfallszustand vorbei. Dann bemerkte ich ein paar unbewohnte Häuser, in deren zerbrochenen Fenster Lumpen gestopft und deren Höfe übersät waren mit Muschelschalen und toten Fischen. Ein- oder zweimal sah ich teilnahmslos wirkende Leute in den kargen Gärten arbeiten oder im Sand des nach Fisch stinkenden Strandes nach Muscheln graben und Gruppen von dreckigen, affengesichtigen Kindern auf den unkrautüberwucherten Schwellen spielen. Auf irgendeine Art wirkten diese Menschen beunruhigender als die trostlosen Gebäude, denn fast alle von ihnen hatten gewisse Eigenheiten in Gesicht und Bewegungen, die mir instinktiv missfielen ohne sie definieren oder verstehen zu können.  Eine Sekunde lang dachte ich, dieser typische Körperbau deutete auf irgendein Bild zurück, das ich einmal gesehen hatte, vielleicht in einem Buch, in einer Situation von außergewöhnlicher Abscheu oder Wehmut, doch diese Pseudoerinnerung verflüchtigte sich schnell wieder.

Als der Bus die tieferliegende Ebene erreichte, begann ich den stetigen Ton eines Wasserfalls durch die unnatürliche Stille wahrzunehmen. Die schiefen, blanken Häuser wurden dichter, säumten beide Seiten der Straße und zeigten eine mehr innerstädtische Richtung als diejenigen, die wir nun hinter uns ließen. Das Panorama voraus hatte sich zu einer Straßenszene zusammengezogen und an manchen stellen konnte ich erkennen, wo Kopfsteinpflaster und Flecken von geziegelten Bürgersteinen einmal existiert hatten. All diese Häuser waren anscheinend verlassen und es gab gelegentliche Lücken wo zusammenfallende Kamine und Kellerwände an Bauwerke erinnerten, die eingestürzt waren. Dies alles wurde durchdrungen vom ekelerregendsten erdenklichen Fischgestank.