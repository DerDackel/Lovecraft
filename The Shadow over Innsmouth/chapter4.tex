Ich kann die Stimmung in der mich diese entsetzliche Begebenheit hinterließ kaum beschreiben --- eine Begebenheit, zugleich wahnsinnig und mitleiderregend, grotesk und furchterregend. Der Junge vom Lebensmittelgeschäft hatte mich darauf vorbereitet, doch hinterließ mich die Realität dadurch nicht weniger fassungslos und verstört. So infantiel die Geschichte auch war, die irrsinnige Ernshaftigkeit und der Schrecken des alten Zadok hatten mir eine sich steigernde Unruhe vermittelt die sich zu meinem bereits vorhandenen Gefühl der Abscheu für diese Stadt und ihren Pesthauch von ungreifbaren Schatten.

Später würde ich die Geschichte überprüfen und einen Kern geschichtlicher Allegorie daraus gewinnen können, doch nun wollte ich sie nur aus meinem Kopf haben. Es war gefährlich spät geworden --- meine Uhr zeigte Viertel nach Sieben und der Bus nach arkham verließ den Markplatz um Acht --- also versuchte ich, meine Gedanken so gut wie möglich in neutrale und zweckmäßige Bahnen zu lenken während ich zügig durch die verlassenen Straßen voller klaffender Dächer und schiefer Häuser in Richtung des Hotels, wo ich meine Reisetasche abgegeben hatte, ging und meinen Bus finden wollte.

Obwohl das goldene Licht des Spätnachmittags den uralten Dächern und baufälligen Kaminen einen Hauch von mystischer Schönheit und Frieden verlieh, konnte ich nicht anders als dann und wann über meine Schulter zu schauen. Ich würde sicherlich froh sein, aus dem übelriechenden und von Angst überschatteten Innsmouth zu verschwinden und wünschte, es gäbe ein anderes Transportmittel als den Bus, den jener finster aussehende Kerl Sargent fuhr. Doch hastete ich nicht zu eilig, dennes gab betrachtenswerte architektonische Details an jeder stillen Ecke und ich konnte, so rechnete ich mir aus, die benötigte Distanz leicht in einer halben Stunde durchmessen.

Die Karte des Jungen aus dem Lebensmittelgeschäft studierend und eine Route suchend, die ich noch nicht durchlaufen hatte, wählte ich die Marsh Street anstatt der State für meinen Weg zum Marktplatz. Nahe der Ecke Fall Street begann ich versprengte Gruppen von verstohlenen Flüsterern zu bemerken und als ich schließlich den Platz erreichte, sah ich, dass fast alle dort bummelnden sich um die Tür des Gilman House versammelt hatten. Es schien als ob zahlreiche hervorgetretene, wässrige, unverwandte Blicke mich sonderbar anstarrten als ich meine Reisetasche in der Lobby einforderte und ich hoffte, dass keine dieser unangenehmen Kreaturen mir im Bus Gesellschaft leisten würden.

Der Bus rasselte ziemlich früh, um kurz vor 8 mit drei Passagieren heran und ein finster schauender Kerl auf dem Bürgersteig murmelte ein paar undeutliche Worte zum Fahrer. Sargent warf einen Postsack und einen Stapel Zeitungen heraus und betrat das Hotel während die Passagiere --- die selben Männer, die ich diesen Morgen in Newburyport hatte ankommen sehen --- schlurften zum Gehsteig und wechselten einige leise, gutturale Worte mit einem Penner in einer Sprache von der ich hätte schwören können, dass es kein Englisch war. Ich bestieg den leeren Bus und nahm den selben Platz, auf dem ich vorhin gesessen hatte, doch ich hatte mich kaum niedergelassen als Sargent wieder erschien und in einer kehligen Stimme von besonderer Widerwärtigkeit zu grummeln begann.

Ich hatte, so schien es, sehr großes Pech. Es gab ein Problem mit dem Motor, trotz der gutgemachten Zeit von Newburyport aus und der Bus könne die Reise nach Arkham nicht vollenden. Nein, er könne weder bis Einbruch der Nacht repariert werden, noch gab es irgendeinen anderes Transportmittel aus Innsmouth heraus, weder nach Arkham, noch sonstwohin. Sargent entschuldigte sich, doch ich würde im Gilham einkehren müssen. Der Empfangschef dort würde mir wahrscheinlich einen guten Preis machen aber es gab keine andere Möglichkeit. Fast verwirrt durch dieses plötzliche Hindernis und den Einbruch der Dunkelheit in dieser verrottenden und halb unbeleuchteten Stadt fürwahr fürchtend, verließ ich den Bus und betrat wieder die Hotellobby, wo der mürrische, sonderbar aussehende Angestellte der Nachtschicht mir sagte, ich könne Zimmer 428 unter dem Dachgeschoss haben --- groß, doch ohne fließend Wasser --- für einen Dollar.

Trotz allem, was ich in Newburyport über dieses Hotel gehört hatte, trug ich mich ins Gästebuch ein, zahlte meinen Dollar, ließ den Angestellten meine Tasche nehmen und folgte dem mürrischen Bediensteten über drei knarrende Treppen an staubigen Korridoren vorbei, die jeglichen Lebens entbehrten. Mein Zimmer, ein trostloses hinteres mit zwei Fenstern und nackten, einfachen Möbeln überblickte einen schmutzigen, ansonsten mit niedrigen, öden Ziegelmauern umgebenen Hof und und gewährte einen Ausblick auf baufällige, sich nach Westen erstreckende Dächer mit sumpfiger Landschaft jenseits davon. Am Ende des Korridors gab es ein Bad --- ein entmutigendes Relikt mit einer uralten Marmorschüssel, Zinnbadewanne, einer schwachen elektrischen Leuchte und modriger Holzverkleidung um alle Armaturen.

Im verbleibenden Tageslicht stieg ich herab zum Platz und schaute mich nach irgendeiner Form von Abendessen um, dabei die befremdlichen Blicke bemerkend, die ich von den unheilsamen Gammlern erhielt. Da der Lebensmittelladen geschlossen hatte, war ich gezwungen das Restaurant aufzusuchen, welches ich vorher gemieden hatte. Eingebeugter, schmalköpfiger Herr mit glotzenden, starren Augen und einem plattnasigen Weib mit unglaublich ungeschickten, plumpen Händen taten dort ihren Dienst. Die Bedienung lief über einen Schalter und es erleichterte mich, dass das meiste aus Dosen und Päckchen serviert wurde. Eine Schale Gemüsesuppe mit ein paar Crackern genügte mir und ich begab mich bald zurück in mein trostloses Zimmer im Gilman, unterwegs eine Abendzeitung und ein abgegriffenes Magazin von einem klapprigen Zeitungsstand bei dem grimmig blickenden Mann am Schalter mitnehmend.

Als die Dämmerung hereinbrach, schaltete ich die schwache Glühbirne über dem billigen Eisengestellbett ein und versuchte, meine begonnene Lektüre fortzusetzen so gut ich konnte. Ich hielt es für ratsam, meinen Verstand vollständig beschäftigt zu halten, da ich nicht über die Abnormalitäten dieses uralten, von Zerstörung überschatteten Städtchens nachdenken wollte, solang ich mich noch innerhalb seiner Grenzen befand. Die irrsinnigen Geschichten, die ich von dem alten Trunkenbold gehört hatte, versprachen keine sonderlich sanften Träume und ich fühlte, dass ich das Bild seiner wilden, wässrigen Augen soweit wie möglich aus meiner Fantasie halten sollte.

Ich durfte auch nicht daran denken, was der Fabrikinspektor dem Fahrkartenverkäufer in Newburyport über das Gilman House und die Stimmen seiner nächtlichen Bewohner erzählt hatte --- nicht daran und auch nicht an das Gesicht unter der Tiara in dem dunklen Kircheneingang. Ein Gesicht, dessen Schrecken mein Bewusstsein nicht erfassen konnte. Es wäre wohl leichter gewesen, meine Gedanken von diesen verstörenden Themen abzulenken, wenn der Raum nicht so schauerlich modrig gewesen wäre. Unter den gegebenen Umständen verschmolz der todbringende Muff auf abscheuliche Weise mit dem allgemeinen Fischgestank der Stadt und lenkte die Fantasie beharrlich auf Tod und Verderbnis.

Eine weitere Sache, die mich störte war die Abwesenheit eines Riegels an der Tür meines Zimmers. Es hatte sich dort einer befunden, wie Spuren klar zeigten, doch gab es Anzeichen, dass er kürzlich entfernt wurde. Zweifellos war er kaputt gegangen wie so viele andere Dinge in diesem heruntergekommenen Gebäude. In meiner Nervosität schaute ich mich um und entdeckte einen Riegel an der Kleiderpresse, der seinen Markierungen nach von der gleichen Größe wie jener, der an der Tür angebracht gewesen war. Um mir etwas Erleichterung von meiner allgemeinen Anspannung zu verschaffen, beschäftigte ich mich damit, dieses Teil mit Hilfe eines Multifunktionswerkzeuges an meinem Schlüsselbund, das einen Schraubenzieher beinhaltete, zu dem freien Platz an der Tür zu versetzen. Der Riegel passte perfekt und ich war ein wenig erleichtert, als ich sicher war, dass ich ihn feste vorschieben konnte, wenn ich zu Bett ging. Nicht, dass ich wirkliche Besorgnis hegte, ihn zu benötigen, doch war mir in einer solchen Umgebung jedes Symbol der Sicherheit willkommen. Die beiden Türen zu den Nebenzimmern hatten angemessene Riegel, welche ich vorschob.

Ich zog mich nicht aus, sondern entschied mich, zu lesen bis ich müde war und mich dann hinzulegen und lediglich Mantel, Kragen und Schuhe abzulegen. Ich nahm eine Taschenlampe aus meiner Reisetasche und steckte sie in meine Hose um die Uhr lesen zu können, falls ich später im Dunkeln aufwachen sollte. Schläfrigkeit überkam mich jedoch nicht und als ich aufhörte, meine Gedanken zu zergliedern, fiel mir zu meiner Beunruhigung auf, dass ich in Wirklichkeit unbewusst nach etwas lauschte --- etwas, vor dem ich mich fürchtete, doch es nicht benennen konnte. Die Geschichte des Inspektors musste schlimmer auf meine Vorstellungskraft eingewirkt haben als ich erwartet hatte. Ich versuchte wieder zu lesen, doch kam damit nicht voran.

Nach einer Weile schien es mir als hörte ich die Treppen und Korridore wiederholt knarren als wie durch Schritte und ich fragte mich, ob die anderen Räume begannen, sich zu füllen. Da waren jedoch keine Stimmen und es war mir als hätte dieses Knarren etwas verstohlenes. Mir gefiel dies nicht und ich erörterte ob ich es überhaupt versuchen sollte, zu schlafen. Diese Stadt hatte einige merkwürdige Einwohner und es sind zweifellos schon ein paar Menschen verschwunden. War dies eines von jenen Gasthäusern wo Reisende für ihr Geld umgebracht wurden? Sicherlich barg ich an mir nicht den Anschein übermäßigen Reichtums. Oder waren die Bürger so verärgert über neugierige Besucher? Hatten meine ersichtlichen Besichtigungen mit ihren häufigen Konsultationen meiner Karte unvorteilhaftes Aufsehen erregt? Mir fiel auf, dass ich in einem hochnervösen Zustand sein musste wenn durch ein zufälliges Knarzen solche Spekulationen in mir hervorrief --- doch ich bereute nichtsdestoweniger, dass ich nicht bewaffnet war.

Endlich, als mich eine Erschöpfung überkam, die nichts an Schläfrigkeit in sich barg, verriegelte ich die neu ausgestattete Tür, schaltete das Licht aus und warf mich auf das harte, unebene Bettzeug, mit Kragen, Schuhen und allem. In der Dunkelheit schien jedes Geräusch verstärkt und eine Welle von gleich doppelt unbehaglichen Gedanken fegte über mich hinweg. Ich bedauerte, dass ich das Licht abgeschaltet hatte, doch war ich zu müde, aufzustehen und es wieder einzuschalten. Dann, nach einer langen, düsteren Zeitspanne, eingeleitet durch neuerliches Knarren von Treppe und Korridor, erklang jenes gedämpfte, unmissverständliche Geräusch, das wie eine unheilvolle Erfüllung aller meiner Befürchtungen schien. Ohne den leisesten Schatten eines Zweifels, versuchte jemand, das Schloss an meiner Zimmertür zu öffnen --- vorsichtig, heimlich, zaghaft --- mit einem Schlüssel.

Mein Aufsehen als ich dieses Zeichen tatsächlicher Gefahr wahrnahm, waren eher weniger stürmisch, wegen meiner vorangegangenen vagen Ängste. Ich war, obgleich ohne eindeutigen Grund, auf der Hut gewesen --- und dies war zu meinem Vorteil in dieser neuen und sehr wirklichen Notlage, welcher Natur sie auch immer sein möge. Nichtsdestotrotz war der Wechsel der Bedrohung von einer unklaren Vorahnung zur unmittelbaren Realität ein tiefgreifender Schock und traf mich mit der Macht eines echten Schlages. Es kam mir nicht einmal in den Sinn, dass das Tasten vielleicht ein schlichtes Versehen sein könnte. Eine bösartige Absicht war alles, an das ich denken konnte und ich verhielt mich totenstill, den nächsten Schritt des möglichen Eindringlings abwartend.

Nach einer Weile endete das vorsichtige Rütteln und ich hörte, wie der Raum zum Norden mit einem Schlüssel betreten wurde. Dann wurde das Schloss der Verbindungstür zu meinem Zimmer probiert. Der Riegel hielt natürlich und ich hörte den Boden knarren, als der Umherstreifende den Raum verließ. Einen Moment später erklang ein weiteres sanftes Rütteln und ich wusste, dass das Zimmer im Süden betreten worden war. Wieder wurde vorsichtig das Schloss der verriegelten Verbindungstür probiert und wieder gab es ein leiser werdendes Knarren. Diesmal ging das Knarzen entlang des Flures und die Treppe herab und ich wusste, der Eindringling hatte den verriegelten Zustand meiner Türen realisiert und hatte seinen Versuch auf kurz oder lang aufgegeben, wie es die Zukunft zeigen sollte.

Die Bereitwilligkeit mit der ich in einen Schlachtplan verfiel bewies, dass ich ihn unterbewusst zurechtgelegt haben musste, eine Bedrohung fürchtend und mögliche Fluchtwege seit Stunden erwägend. Von Anfang an spürte ich, dass der ungesehene Fremde eine Gefahr darstellte, auf die man nicht treffen oder es mit ihr zu tun bekommen, sondern vor der man so hastig wie möglich fliehen sollte. Das einzige was zu tun war, war so schnell wie möglich lebend aus dem Hotel zu entkommen und das  auf einem anderen Weg als durch das Treppenhaus und die Lobby.

Leise aufstehend und meine Taschenlampe auf den Lichtschalter scheinend, suchte ich die Birne über meinem Bett einzuschalten um ein paar Habseligkeiten für eine schnelle Flucht ohne meine Reisetasche zusammenzusuchen und einzustecken. Es passierte jedoch nichts und ich musste feststellen, dass der Strom abgestellt worden war. Offenbar war hier ein groß angelegtes, rätselhaftes, böses Uhrwerk im Gange --- doch mehr konnte ich nicht ahnen. Als ich grübelnd dastand, die Hand noch auf dem nutzlosen Schalter, hörte ich ein gedämpftes Knarren im Stockwerk unter mir und glaubte, mit Mühe und Not Stimmen im Gespräch miteinander unterscheiden zu können. Einen Moment später war ich mir weniger sicher ob es sich bei den tieferen Lauten wirklich um Stimmen handelte, da das scheinbar heisere Gebell und dumpfes Quaken sowenig von bekannter menschlicher Sprache bargen. Dann dachte ich mit erneuter Macht daran, was der Fabrikinspektor in der Nacht in diesem zerfallenden, pesterfüllten Gebäude.

Meine Taschen mit Hilfe der Taschenlampe gefüllt, setzte ich meinen Hut auf und schlich auf Zehenspitzen an die Fenster um meine Chancen für einen Abstieg zu erkunden. Trotz staatlicher Sicherheitsvorschriften, gab es an der Seite des Hotels keine Feuerleiter und ich sah, dass von meinen Fenstern aus nur ein steiler Sturz über drei Stockwerke möglich war. Rechts und links grenzten jedoch ein paar uralte backsteinerne Geschäftsbauten an das Hotel, deren schräge Dächer in annehmbarer Sprungweite aus der Höhe des vierten Stocks lagen. Um eine dieser Gebäudezeilen zu erreichen, würde ich ein ein Zimmer zwei Türen weit von meinem --- im einen Fall im Norden, im anderen im Süden gelangen müssen --- und mein Verstand machte sich sofort daran, zu berechnen welche Chancen ich hätte, den Sprung zu schaffen.

Ich konnte, so entschied ich, es nicht riskieren, auf den Korridor hervorzutreten, wo meine Schritte sicherlich gehört werden würden und wo die Hindernisse, das gewünschte Zimmer zu betreten unüberwindbar sein würden. Mein Fortkommen, sollte ich es überhaupt schaffen, würde durch die weniger solide gebauten Zwischentüren, die die Räume verbanden, führen müssen, deren Schlösser und Riegel ich mit Gewalt aufzuzwingen haben würde indem ich meine Schulter als Rammbock gebrauchte wo immer sie sich mir entgegensetzten. Dies würde, so dachte ich, durch die wacklige Natur des Hauses und seiner Einbauten möglich sein, doch mir wurde klar, dass ich dies nicht geräuschlos durchführen konnte. Ich würde auf bloße Geschwindigkeit zählen müssen und auf die Chance, ein Fenster zu erreichen, bevor irgendwelche feindseligen Kräfte koordiniert genug agieren würden um die richtige Tür vor mir mit einem Schlüssel zu öffnen. Meine eigene Zimmertür verstärkte ich indem ich die Kommode Stück für Stück dagegen schob um ein Minimum an Geräuschen zu verursachen.

Ich erkannte, dass meine Chancen sehr mager aussahen und war auf jede Katastrophe gefasst. Auch ein anderes Dach zu erreichen würde das Problem nicht lösen, denn es würde dann immer noch die Aufgabe bleiben, den Boden zu erreichen und aus der Stadt zu fliehen. Eine Sache die zu meinen Gunsten stand war der menschenleere und verfallene Zustand der angrenzenden Gebäude und die Anzahl an Dachluken, die in jeder Reihe schwarz aufklafften.

Der Karte des Lebensmittelverkäufers entnehmend, dass die beste Route aus der Stadt südwärts führte, blickte ich zuerst an die Verbindungstür an der Südseite des Raumes. Sie war so konzipiert, dass sie in meine Richtung öffnete und so sah ich --- nachdem ich den Riegel zurückgeschoben und weitere Befestigungen vorgefunden hatte --- dass sie sich nicht zum Aufbrechen anbot. Dementsprechend gab ich es als Fluchtweg auf und bewegte vorsichtig das Bettgestell dagegen um jeden Angriff, der später vom Nebenraum aus darauf ausgeübt werden mochte zu erschweren. Die Tür zum Norden öffnete sich weg von mir und diese --- obwohl eine Überprüfung zeigte, dass sie von der anderen Seite verschlossen und verriegelt war --- musste, so wusste ich, meine Route sein. Wenn ich die Dächer der Gebäude in der Paine Street erreichen und erfolgreich ins Erdgeschoss absteigen könnte, würde ich wahrscheinlich durch den Hof und die angrenzenden oder gegenüberliegenden Bauten zur Washington oder zur Bates rennen --- oder sonst auf die Paine heraustreten und südwärts um die Ecke zur Washington. Jedenfalls beabsichtigte ich, irgendwie die Washington zu erreichen und schnell aus der Gegend des Marktplatzes zu gelangen. Ich würde es bevorzugen, die Paine zu meiden, da die Feuerwache dort die ganze Nacht besetzt sein könnte.

Während ich darüber nachdachte, schaute ich herab auf das verwahrloste Meer verrottender Dächer unter mir, nun erhellt durch einen nicht mehr ganz vollen Mond. Zur Rechten teilte die schwarze Spalte der Flussklamm das Panorama; verlassene Fabriken und der Bahnhof an ihren Wänden klammernd wie Seepocken. Jenseits davon führten die verrostete Eisenbahn und die Straße nach Rowley durch eine flache, sumpfige Landschaft. durchzogen von kleinen Inseln von höhergelegenem und trockenerem, von Gestrüpp bewachsenem Land. Zur Linken lag die von Bächen durchzogene Landschaft näher, die schmale Straße nach Ipswich weiß im Mondlicht glänzend. Ich konnte von meiner Seite des Hotels aus die südwärts führende Route nach Arkham, die ich gewählt hatte nicht sehen.

Ich spekulierte unentschlossen, ob ich mir besser die Nordtür vornehmen sollte und wie ich dies am leisesten bewerkstelligen könnte, als ich bemerkte, dass die schwachen Geräusche von unten neuem und schwererem Knarren der Treppen gewichen waren. Eine flackernde Zunge aus Licht zeigte sich unter dem Querbalken und die Bretter im Korridor ächzten unter schwerer Last. Gedämpfte Laute möglicherweise gesprochenen Urpsungs kamen näher und schließlich schlug ein Stoß an die Außenseite meiner Tür.

Einen Moment lang hielt ich nur meinen Atem an und wartete ab. Ewigkeiten schienen zu verstreichen und der Fischgestank in meiner Umgebung plötzlich und eindrucksvoll anzuwachsen. Dann wiederholten sich die Schläge --- ununterbrochen und mit wachsender Beharrlichkeit. Ich wusste, dass die Zeit zu Handeln gekommen war und schob den Riegel der Nordtür zurück, mich auf die Aufgabe vorbereitend, sie aufzubrechen. Die Stöße wurden lauter und ich hoffte, dass ihre Lautstärke das Geräusch meiner Bemühungen überdecken würde. Endlich begann ich meinen Versuch; ich warf mich immer und immer wieder mit meiner linken Schulter gegen die dünne Holzverkleidung, ohne Rücksicht auf Schock oder Schmerz. Die Tür widerstand noch mehr als ich erwartet hatte, doch gab ich nicht nach. Währenddessen verstärkte sich der Lärm an der Außentür weiter.

Endlich gab die Verbindungstür nach, doch mit einem solchen Krachen, dass ich wusste, es musste auch von denen draußen gehört worden sein. Sofort wurde aus dem Pochen von außen ein heftiges Rammen, während Schlüssel unheilvoll in den Zimmertüren der Räume zu beiden Seiten von mir rasselten. Durch die neugeschaffene Verbindung hastend, schaffte ich es, die nördliche Zimmertüre zu versperren bevor das Schloss geöffnet werden konnte, doch noch während ich das tat, hörte ich, wie die Türe des dritten Raumes --- desjenigen dessen Fenster ich zu erreichen gehofft hatte mit einem Schlüssel aufgeschlossen wurde.

Für einen Augenblick fühlte ich absolute Verzweiflung, da ich in einer Kammer ohne Ausweg durchs Fenster gefangen schien. Eine Woge von fast abnormem Schrecken überspülte mich und richtete sich mit einer furchtbaren doch unerklärlichen Außergewöhnlichkeit auf die im Licht der Taschenlampe erspähten Handabdrücke im Staub, die der Eindringling welcher vorhin meine Tür aus diesem Raum probiert hatte, hinterlassen hatte. Dann rannte ich in einem benommenen Automatismus, der ob jeder Hoffnungslosigkeit bestand, zur nächsten Verbindungstür und drückte blind dagegen in meiner Bemühung hindurch zu gelangen und --- sofern die Riegel wie im zweiten Raum zu meinem Glück intakt sein sollten --- die Zimmertür zu verriegeln bevor sie von außen geöffnet werden konnte.

Durch schieres Glück erhielt ich eine Atempause --- denn die Verbindungstür vor mir war nicht nur unverschlossen, sondern stand dazu noch offen. Innerhalb einer Sekunde war ich hindurch und war mein rechtes Knie und Schulter gegen die Zimmertür, die sich sichtbar nach innen öffnete. Mein Rick schien den Öffnenden zu überraschenden, denn die Tür schloss sich als ich drückte, so dass ich den in gutem Zustand befindlichen Riegel vorschieben konnte, so wie ich es mit der anderen Tür getan hatte. Als ich diesen Aufschub erhielt hörte ich, wie die Schläge gegen die anderen beiden Türen nachließen, während ein konfuses Poltern von der Verbindungstür, die ich mit dem Bettgestell versperrt hatte, ertönte. Offenbar hatte die Masse meiner Angreifer den südlichen Raum betreten und sammelte sich zu einer Attacke von der Seite. Doch im selben Moment erklang ein Schlüssel in der nächsten Tür von Norden und ich wusste, dass mir eine andere Gefahr bevorstand.

Die nördliche Verbindungstür stand offen, doch ich hatte keine Zeit über das sich bereits öffnenede Schloss zum Flur nachzudenken. Alles, was ich tun konnte war, die Verbindungstür zu schließen und zu verriegeln, eben so ihr Gegenstück auf der anderen, ein Bettgestell gegen die eine und eine Kommode gegen die andere zu schieben und einen  Waschtisch vor die Zimmertür zu schieben. Ich musste, so wie ich sah, diesen notdürftigen Barrieren vertrauen, mich so lange zu schützen, bis ich aus dem Fenster und auf das Dach des Paine Street Blocks gelangen konnte. Doch selbst in diesem Moment bestand meine bedeutendster Schrecken in etwas jenseits der unmittelbaren Schwäche meiner Verteidigung. Ich erschauderte, weil nicht ein einziger meiner Verfolger trotz einigen abscheulichen Keuchens, Grunzens und unterdrückten Gebells in seltsamen Abständen, auch nur einen ungedämpften oder verständlichen menschlichen Laut von sich gab.

Während ich die Möbel verschob und zu den Fenstern rannte, hörte ich ein schreckliches Trippeln entlang des Korridors in Richtung des Zimmers nördlich von mir. Offenbar waren die meisten meiner Gegner dabei, sich auf die schwache Zwischentür zu konzentrieren von der sie wussten, dass sie direkt zu mir führte. Draußen spielte der Mond auf dem Dachfirst des Blockes unter mir und ich sah, dass der Sprung sehr gefährlich werden würde, da ich auf einer sehr steilen Oberfläche landen musste.

Die Umstände überblickend, wählte ich das südliche der beiden Fenster als meinen Fluchtweg und plante, auf der inneren Dachschräge zu landen um danach das nächste Dachfenster zu erreichen. Einmal in einem der verfallenen Ziegelgebäude angelangt, würde ich mit Verfolgung rechnen müssen, doch ich hoffte, rasch hinabzusteigen und mich entlang der klaffenden Eingänge zu verstecken, um schließlich Washington Street zu erreichen und der Stadt nach Süden zu entfliehen.

Das Trampeln an der nördlichen Verbindungstür war nun fürchterlich und ich sah, wie die dünne Holzverkleidung zu splittern begann. Offensichtlich hatten die Belagerer ein schweres Objekt als Rammbock ins Spiel gebracht. Das Bettgestell jedoch hielt immer noch stand, so dass ich zumindest eine blasse Chance hatte, meine Flucht voranzutreiben. Als ich das Fenster öffnete, bemerkte ich, dass es von schweren Veloursgardinen gesäumt war, die an Messingringen von einer Stange hingen und auch, dass es außen einen weit hervorstehenden Fang für die Fensterläden gab. Einen möglichen Weg erkennend, den gefährlichen Sprung zu vermeiden, riss ich and den Behängen und holte sie herab, samt Stange und allem anderen, hing daraufhin schnell zwei der Ringe in den Fang und warf die Gardinen nach draußen. Die schweren Vorhänge reichten ganz bis zum angrenzenden Dach und ich überprüfte, dass die Ringe und der Fang mein Gewicht tragen würden. So verließ ich, aus dem Fenster die improvisierte Strickleiter herunterkletternd den morbiden und von Schrecken heimgesuchten Bau des Gilman House für immer.

Ich landete sicher auf den losen Schindeln des steilen Daches und schaffte es, das klaffende, schwarze Dachfenster zu erreichen ohne auszurutschen. Ich stellte fest, dass es noch immer dunkel war, jedoch weit über die bröckelnden Schornsteine hinaus unheilvolle Lichter in der Ordenshalle des Dagon, der Baptisten- und der Gemeindekirche, an die ich schaudernd zurückdachte, funkelten. Es hatte den Anschein erweckt als befinde sich niemand im Hof unter mir und ich hoffte, dass die Möglichkeit sich bieten würde, zu verschwinden bevor ein Großalarm ausgerufen wurde. Meine Taschenlampe in das Oberlicht leuchtend, sah ich, dass es keine Leiter nach unten gab. Die Entfernung war jedoch nur kurz und so kletterte ich über die Kanteund ließ mich fallen; auf einen staubigen Boden, der übersät war mit zerfallenden Kisten und Fässern.

Dieser Ort sah gräulich aus, doch ich befand jenseits solcher Eindrücke und machte mich sofort zur von meiner Lampe offenbarten Treppe --- nach einem flüchtigen Blick auf meine Uhr, die mir Zwei Uhr morgens anzeigte. Die Stufen knarzten, doch schienen sie hinreichend stabil zu sein und ich eilte hinunter vorbei an einem scheunenartigen ersten Stock ins Erdgeschoss. Ich befand mich in vollkommener Verlassenheit und nur das Echo erwiderte meine Schritte. Bald erreichte ich den unteren Flur, an dessen Ende ich ein schwach leuchtendes Rechteck erkannte, dass den zerstörten Ausgang zur Paine Street markierte. In die andere Richtung fand ich die Hintertür ebenfalls offen und flitzte hinaus, über fünf steinerne Stufen auf das grasumwachsene Kopfsteinpflaster des Hofes.

Das Mondlicht reichte nicht hier herunter, doch konnte ich meinen Weg ohne Hilfe der Taschenlampe erspähen. Einige der Fenster des Gilman House leuchteten schwach und ich wähnte, darin konfuse Laute zu vernehmen. Vorsichtig zur Washington Street laufend bemerkte ich mehrere offene Hauseingänge und wählte den ersten für meine Fluchtroute. Der Flur dahinter war schwarz und als ich das andere Ende erreichte, stellte ich fest, dass die Tür zur Straße verschlossen und unbeweglich verklemmt war. Ich tastete meinen Weg zurück zum Hof, doch hielt ich inne als ich den Durchgang erreichte.

Denn aus der geöffneten Tür des Gilman House strömte eine große Schar verdächtiger Formen --- mit in der Dunkelheit tanzenden Laternen und abscheulichen quakenden Stimmen, die tiefe Schreie austauschten, die sicher nicht Englisch waren. Die Gestalten bewegten sich ungewiss und ich stellte zu meiner Erleichterung fest, dass sie nicht wussten, wohin ich geflohen war; doch trotz allem schickten sie einen Schauder des Entsetzen über meinen Rücken. Ihre Züge waren nicht zu erkennen, doch ihr kriechender, schlurfender Gang war widerwärtig abstoßend. Und das Allerschlimmste war, als ich eine Gestalt bemerkte, die fremdartig gekleidet und unverkennbar von einer schlanken Tiara gekrönt war, deren Formgebung mir nur allzu bekannt war. Als die Gestalten sich über den Hof verteilten, verstärkte sich meine Angst. Was, wenn ich keinen Ausgang zur Straße aus diesem Gebäude fand? Der Fischgestank war abscheulich und ich fragte mich, wie ich ihn aushalten konnte ohne einen Ohnmachtsanfall zu erleiden. Mich wieder zur Straße hin tastend, öffnete ich eine Seitentür des Flures und fand einen leeren Raum mit fest verschlossenen, doch rahmenlosen Fenstern. Im Schein meiner Taschenlampe umhertastend fand ich heraus, dass die Läden zu öffnen waren und einen Moment später war ich herausgeklettert und verschloss vorsichtig die Öffnung wieder auf die gleiche Manier.

Ich befand mich nun auf der Washington Street und sah für den Augenblick weder Leben, noch irgendwelches Licht mit Ausnahme des Mondes. Aus mehreren Richtungen in der Ferne jedoch, konnte ich den Klang heiserer Stimmen und von Fußstapfen vernehmen. Ich hatte schlicht keine Zeit zu verlieren. Ich war mir der Himmelsrichtungen bewusst und froh, dass die Straßenbeleuchtung abgeschaltet war, wie es an Nächten mit hellem Mondschein oft der Fall in armen, ländlichen Gegenden war. Einige der Laute kamen aus dem Süden, doch behielt ich meinen Plan bei, in diese Richtung zu entkommen. Es würde, soweit war ich mir sicher, zahlreiche verlassene Eingänge geben, in denen ich mich verstecken konnte, sofern ich auf irgendeine Person oder Grupe treffen sollte, die nach Verfolgern aussah.

Ich lief schnell, leise und hielt mich an die verfallenen Häuser. Obwohl hutlos und zerzaust, sah ich nicht besonders auffällig aus und hatte eine gute Chance, unbemerkt zu passieren, sollte ich gezwungen sein, auf einen zufälligen Fußgänger zu treffen. An der Bates Street bog ich in einen klaffenden Vorhof als zwei watschelnde Gestalten meinen Weg kreuzten, doch war ich bald wieder unterwegs und näherte mich dem offenen Platz an dem die Eliot Street die Washington schräg kreuzt. Obwohl ich diesen Ort nie gesehen hatte, hatte er auf der Karte des Lebensmitteljungen gefährlich ausgesehen, da das Mondlicht hier freies Spiel hatte. Es hatte keinen Zweck, ihm auszuweichen, denn jede andere Route beinhaltete Umleitungen von womöglich verheerender Sichtbarkeit und Verzögerung. Die einzige Möglichkeit bestand darin, den Platz mutig und offen zu überqueren, dabei das typische Watscheln der Innsmouther imitierend, so gut ich konnte und darauf zu hoffen, dass niemand --- oder zumindest keiner meiner Verfolger dort sein würde.

Davon, wie stark die Verfolgung organisiert war --- und was ihr tatsächlicher Zweck sein mochte, konnte ich mir keinen Begriff machen. Die Stadt schien ungewöhnlich lebendig, doch ich schätzte, dass die Kunde von meiner Flucht aus dem Gilman sich noch nicht verbreitet hatte. Ich würde natürlich bald von der Washington auf eine andere Straße nach Süden wechseln müssen, denn die Gesellschaft aus dem Hotel würde zweifelsohne hinter mir her sein. Ich musste Abdrücke im Staub jenes alten Gebäudes hinterlassen haben, die offenbarten, wie ich zur Straße herab gelangt war.

Der offene Platz war, wie erwartet, stark vom Mondlicht erhellt und ich sah die Reste eines parkartigen, eines eisenumzäunten Grüns in seiner Mitte. Glücklicherweise war niemand zugegen, obwohl sich eine Art merkwürdiges Brummen oder Dröhnen aus der Richtung des Marktplatzes aufbaute. South Street war sehr breit und führte an einem kleinen Abhang hinunter direkt ans Ufer und gewährte dort einen Ausblick auf die See --- und ich hoffte, dass von dort niemand aufblickte, während ich den Platz im hellen Mondlicht überquerte.

Mit ungebrochenem Fortschritt und ohne neuerliche Geräusche, die mich darauf hingewiesen hätten, dass ich entdeckt worden war, mäßigte ich mein Tempo für eine Sekunde um den Ausblick auf die See auf mich wirken zu lassen, prachtvoll im glühenden Mondlicht am Ende der Straße. Weit draußen jenseits der Mole war die trübe, dunkle Linie des Devil Reef und als ich es erblickte konnte ich nicht anders, als an all jene hässlichen Legenden, die ich in den letzten vierunddreißig Stunden gehört hatte zu denken --- Legenden die diesen zackigen Felsen als ein wahrhaftiges Tor zu Gefilden unergründeten Schreckens und undenkbarer Abnormität darstellten.

Dann, ohne Vorwarnung, bemerkte ich periodische Lichtblitze auf dem entfernten Riff. Sie waren deutlich und unverkennbar und weckten in meinem Geist eine Blinde Furcht jenseits aller rationalen Maßstäbe. Meine Muskeln spannten sich zur panischen Flucht und wurden nur durch unbewusste Vorsicht und eine halb-hypnotische Faszination gehalten. Und um alles noch schlimmer zu macchen, blitzte nun auch aus der hohen Kuppel des Gilman House, welches im Nordosten hinter mir aufragte eine Reihe vergleichbarer, doch anders aufeinanderfolgender Schimmer die nicht weniger als ein Antwortsignal gewesen sein konnten.

Meine Muskeln wieder kontrollierend und erneut begreifend, wie ausgesprochen sichtbar ich stand, nahm ich meinen strammeren und verstellt watschelnden Schritt wieder auf, doch behielt ich meinen Blick auf dem höllischen und unheilvollen Riff so lange die Öffnung der South Street mir einen BLick zur See gewährte. Was die ganze Prozedur bedeuten sollte, konnte ich mir nicht ausmalen, außer dass es sich um ein fremdartiges Ritual im Zusammenhang mit Devil Reef handelte, oder dass eine Gruppe von einem Schiff aus auf dem finsteren Fels gelandet sei. Ich beugte mich nun nach links um das verfallene Grün, noch immer in Richtung des Ozeans starrend, der im gespenstischen Mondlicht des Sommers funkelte und das kryptische Blinken jener namenlosen, unerklärlichen Leuchtfeuer beobachtend.

Es war genau dann, dass der furchtbarste Eindruck all dessen in mir aufkam --- der Eindruck, der die letzten Überbleibsel von Selbstbeherrschung in mir zerstörte und mich verzweifelt nach Süden, an gähnend schwarzen Eingängen und fischartig starrenden Fenstern jener verlassenen Straße der Alpträume. Denn bei genauerem Blick sah ich, dass die mondbeschienenen Wasser zwischen dem Riff und der Küste alles andere als leer waren. Sie schienen lebendig vor einer wummelnden Horde von Gestalten, die einwärts zur Stadt schwammen und sogar auf meine große Entfernung und in einem einzigen Moment der Wahrnehmung konnte ich erkennen, dass ihre sich auf und ab bewegenden Köpfe und ihre strampelnden Arme fremd und auf eine Weise abnormal waren, der sich kaum ausdrücken oder bewusst formulieren ließ.

      It was then that the most horrible impression of all was borne in upon me—the impression which destroyed my last vestige of self-control and set me running frantically southward past the yawning black doorways and fishily staring windows of that deserted nightmare street. For at a closer glance I saw that the moonlit waters between the reef and the shore were far from empty. They were alive with a teeming horde of shapes swimming inward toward the town; and even at my vast distance and in my single moment of perception I could tell that the bobbing heads and flailing arms were alien and aberrant in a way scarcely to be expressed or consciously formulated.
