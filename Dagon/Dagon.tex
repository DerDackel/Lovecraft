\documentclass[a4paper]{memoir}
\usepackage[ngerman]{babel}
\usepackage{textcomp}
\usepackage[cm-default]{fontspec}
\usepackage{xunicode}
\usepackage{xltxtra}
\setromanfont[Mapping=tex-text]{Linux Libertine O} %roman font


\begin{document}
\title{Dagon}
\author{H.P. Lovecraft\\
		Sebastian Jackel (Übersetzung)}
\date{}
\maketitle

\textit{Dieser Text steht unter einer Creative Commons CC-BY-SA 3.0 Lizenz
(siehe\\ creativecommons.org/licenses/by-sa/3.0). Das Englische Original von H.P. Lovecraft ist Teil der Public Domain. Die dieser Übersetzung zugrunde liegende Originalfassung kann unter hplovecraft.com nachgelesen werden.}

\vspace{12pt}

Ich schreibe diese Zeilen unter beträchtlicher seelischer Belastung, denn nach der heutigen Nacht werde ich nicht mehr sein. Mittellos und am Ende meiner Vorräte der Droge, die allein mein Leben erträglich macht, kann ich die Folter nicht länger aushalten und werde mich vom Fenster dieser Dachkammer auf die schmutzige Straße herunter stürzen. Schließt nicht aus meiner Existenz in der Knechtschaft des Morphiums, dass ich ein Schwächling oder Degenerierter bin. Wenn Ihr diese hastig gekritzelten Zeilen gelesen habt, vermögt ihr vielleicht zu erahnen, doch niemals vollständig zu begreifen, warum ich nach Vergessen oder Tod verlange.

Es war in einer der offensten und am seltensten befahrenen Gegenden des Pazifiks, dass das Paketschiff, auf dem ich Ladungsoffizier war, zur deutschen Kriegsbeute wurde. Der Große Krieg stand noch ganz am Anfang seines Verlaufs und die Seemacht der Deutschen war noch nicht komplett ihrer späteren Schande verfallen, so dass unser Schiff zur rechtmäßigen Beute wurde, während wir als ihre Besatzung mit allem Anstand und Rücksicht behandelt wurden, die uns als Gefangene auf See zustand. In der Tat war die Disziplin unserer Wärter so liberal, dass ich es nach fünf Tagen schaffte, allein in einem kleinen Boot mit genug Wasser und Proviant für lange Zeit zu entkommen.

Als ich mich endlich treibend und frei wiederfand, hatte ich kaum eine Vorstellung meiner Umgebung. Noch nie ein guter Navigator, konnte ich nur grob anhand von Sonne und Sternen erahnen, dass ich mich leicht südlich des Äquators befand. Den Längengrad vermochte ich nicht zu bestimmen und keine Insel oder Küstenlinie war in Sicht. Das Wetter blieb gut und für ungezählte Tage trieb ich ziellos unter der brennenden Sonne, wartend auf ein vorbeifahrendes Schiff oder bewohnbares Land. Doch weder Schiff noch Land tauchten auf und ich begann zu verzweifeln gegenüber meiner Einsamkeit in den wogenden Weiten ungebrochenen Blaus.

Die Veränderung ereignete sich, während ich schlief. Ihre Einzelheiten werde ich niemals erfahren, denn mein Schlaf, obwohl unruhig und von Träumen heimgesucht, war ununterbrochen. Als ich endlich erwachte, fand ich mich halb verschlungen von einer schleimigen Fläche aus höllischem, schwarzem Morast, der sich um mich in gleichförmigen Wellen erstreckte, soweit ich sehen konnte und in welchem mein Boot in einiger Entfernung auf Grund lag.

Obwohl man sich vorstellen mag, dass mein erster Eindruck einer von Erstaunen über eine solch außerordentliche Transformation der Landschaft gewesen wäre, war ich in Wirklichkeit eher entsetzt als erstaunt, denn die Luft und der verrottende Boden bargen eine sinistre Qualität in sich, die mich bis ins Innerste erschaudern ließ. Die Gegend war verdorben von den Kadavern verwesender Fische undanderer, weniger beschreiblicherer Dinge, die ich aus dem garstigen Schlamm der endlosen Ebene herausragen sah. Ich sollte wahrscheinlich nicht darauf hoffen, die unaussprechliche Abscheulichkeit, die absoluter Stille und karger Unermesslichkeit innewohnen in Worten beschreiben zu können. Da war nichts in Hörweite und nichts in Sicht bis auf einen gewaltiges Areal von schwarzem Schleim, doch die Vollkommenheit der Totenstille und die Monotonie der Landschaft erdrückten mich mit widerlicher Furcht.

Die Sonne brannte an einem Himmel, der mir in seiner wolkenlosen Grausamkeit fast schwarz erschien; als ob er den tiefschwarzen Sumpf unter meinen Füßen reflektierte. Als ich in das gestrandete Boot krabbelte, wurde mir klar, dass nur eine Theorie meine Lage zu erklären vermochte. Durch einen beispiellosen vulkanischen Aufruhr musste ein Teil des Meeresbodens an die Oberfläche aufgeworfen worden sein, Regionen freilegend, die für unzählige Millionen von Jahren unter unergründlichen nassen Tiefen verborgen waren. So groß war die Ausdehnung des neuen Landes, das sich unter mir erhoben hatte, dass ich nicht das leiseste Geräusch des wogenden Ozeans vernehmen konnte, so sehr sich meine Ohren auch anstrengten. Noch waren da irgendwelche Meeresvögel um sich an den toten Dingen zu weiden.

Mehrere Stunden lang saß ich nachdenkend oder grübelnd im Boot, das auf der Seite lag und etwas Schatten spendete, während die Sonne sich über den Himmel bewegte. Während der Tag verging, verlor der Boden etwas von seiner Klebrigkeit und es schien wahrscheinlich, dass er in Kürze trocken genug sein würde um darauf zu wandern. In jener Nacht schlief ich kaum und am nächsten Tag fertigte ich mir ein Bündel mit Essen und Wasser darin, in Vorbereitung auf eine Reise überland auf der Suche nach der verschwundenen See und möglicher Rettung.

Am dritten Morgen fand ich die Erde trocken genug um darauf mit Leichtigkeit zu gehen. Der Fischgestank war unerträglich, doch ich war zu sehr mit schwerwiegenderen Dingen beschäftigt um ein solch geringes Übel zu beachten und machte mich mutig auf zu einem unbekannten Ziel. Den ganzen Tag kam ich westwärts voran, geleitet von einer kleinen Anhöhe, die sich über jede andere Erhebung in dieser hügeligen Wüste erhob. In der Nacht kampierte ich und am folgenden Tag wanderte ich weiter in Richtung der Anhöhe, obwohl sie mir kaum näher erschien als wenn ich es zuerst erspäht hatte. Am vierten Abend kam ich am Fuß der Erhebung an, die sich als viel höher herausstellte, als sie aus der Ferne gewirkt hatte, mit einem dazwischenfahrenden Tal, das sie stärker von der sonstigen Oberfläche absetzte. Zu erschöpft um hinaufzusteigen, schlief ich im Schatten des Hügels ein.

Ich weiß nicht, warum meine Träume in dieser Nacht so wild waren, doch ehe der abnehmende, phantastische Dreiviertelmond sich weit über die östliche Ebene erhoben hatte, war ich erwacht, in kaltem Schweiß gebadet und entschlossen, nicht weiter zu schlafen. Die Visionen die mir widerfahren waren, waren zuviel für mich um sie noch einmal zu ertragen. Und im Leuchten des Mondes erkannte ich, wie töricht ich gewesen war, am Tage zu reisen. Ohne die Grelle der sengenden Sonne hätte mich meine Reise viel weniger Kraft gekostet. In der Tat fühlte ich mich nun sehr wohl fähig, den Aufstieg, der mich vor Sonnenuntergang abgeschreckt hatte, zu bewältigen. Mein Bündel geschultert, brach ich zum Gipfel der Anhöhe auf.

Ich erwähnte bereits, dass die ununterbrochene Monotonie der hügeligen Ebene ein Quell vager Furcht für mich war, doch ich glaube, mein Grauen war noch größer als ich die Spitze der Erhebung erklommen hatte und auf der anderen Seite in eine unermessliche Grube oder einen Canyon schaute, über dessen schwarze Winkel der Mond sich noch nicht hoch genug erhoben hatte um sie zu erhellen. Ich fühlte mich als stünde ich am Ende der Welt, über den Rand ins unergründliche Chaos ewiger Nacht starrend. Durch meinen Schrecken rannen sonderbare Erinnerungen an \textit{Das verlorene Paradies} und Satans abscheulichen Aufstieg durch die formlosen Reiche der Dunkelheit.

Als der Mond am Himmel höher aufstieg begann ich zu erkennen, dass die Hänge des Tals nicht ganz so lotrecht verliefen, wie ich vermutet hatte. Vorsprünge und Felszungen gewährten bequemen Halt für einen Abstieg, während der Abhang nach einem Sturz von einigen hundert Fuß in Stufen überging. Gedrängt durch einen Impuls den ich nicht eindeutig zuordnen kann, kletterte ich mühsam über die Felsen herab und stand auf dem flacheren Abhang darunter, in die stygischen Tiefen starrend, in die noch kein Licht vorgedrungen war.

Auf einmal wurde meine Aufmerksamkeit von einem riesigen, einzigartigen Objekt  auf dem gegenüberliegenden Hang eingefangen, das etwa hundert Schritte vor mir steil aufragte; ein Objekt, das im nun darauf fallenden Licht des aufsteigenden Mondes weißlich glänzte. Ich versicherte mich bald, dass es sich lediglich um einen riesigen Felsbrocken handelte, doch war ich mir eines gewissen Eindrucks bewusst, dass seine Kontur und Position nicht vollständig das Werk der Natur waren. Eine nähere Prüfung erfüllte mich mit Eindrücken, die ich nicht auszudrücken vermag, denn trotz seiner enormen Größe und seinem Standpunkt in einem Abgrund, der am Meeresboden geklafft hatte seit die Welt jung war, erkannte ich ohne jeden Zweifel, dass das sonderbare Objekt ein wohlgeformter Monolith war, dessen gewaltige Masse Bearbeitung und vielleicht Verehrung von lebenden und denkenden Kreaturen erfahren hatte.

Benommen und verängstigt, doch nicht ohne eine Art wissenschaftliche Neugier oder die Freude des Archäologen, betrachtete ich meine Umgebung näher. Der Mond stand nun fast im Zenit und schien unheimlich und klar über den gewaltigen Steilhängen, die die Schlucht säumten und offenbarte, dass ein entlegenes Gewässer am Grund floß, sich in beide Richtungen außer Sicht wand und fast zu meinen Füßen plätscherte während ich am Hang stand. Jenseits der Spalte wuschen die Wellen den Fuß des zyklopischen Monolithen, an dessen Oberfläche ich nun sowohl Inschriften als auch grobe Bildhauereien erkannte. Das Geschriebene stand in einem System von Hieroglyphen, die mir unbekannt waren, die wie nichts aussahen, das ich je in Büchern gesehen hatte und die zum größten Teil aus stilisierten aquatischen Symbolen, wie Fischen, Aalen, Kraken, Krustentieren, Mollusken, Walen und ähnlichem bestanden. Mehrere Zeichen repräsentierten offensichtlich Meereslebewesen, die der modernen Welt unbekannt waren, doch deren verwesende Formen ich auf der aus dem Ozean gestiegenen Ebene erkannt hatte.

Es waren jedoch die Bildschnitzereien die mich am meisten faszinierten. Aufgrund ihrer enormen Größe war eine Reihe von Basreliefs klar, deren Themen den Neid eines Doré erregt hätten, über das Wasser hinweg sichtbar. Ich glaube, sie sollten Menschen darstellen --- zumindest eine Art Menschen, obwohl die Kreaturen sich wie Fische vergnügend in den Wassern einer Meeresgrotte dargestellt, oder huldigend an einer Art monolithischem Schrein abgebildet waren, der sich ebenfalls unter den Wellen zu befinden schien. Von ihren Gesichtern und Formen wage ich nicht im Detail zu sprechen, denn die bloße Erinnerung daran bringt mich der Ohnmacht nahe. Grotesk jenseits der Vorstellungskraft eines Poe oder eines Bulwer, waren sie abscheulich menschlich in ihren Konturen trotz Schwimmhäuten an Händen und Füßen, furchtbar breiter, hängender Lippen, glasiger, hervortretender Augen und anderer Merkmale, die ich mir ungern ins Gedächtnis zurückrufe. Kurioserweise schienen sie völlig außer Proportion zu ihrem Hintergrund gemeißelt zu sein, denn eine der Kreaturen war beim Töten eines Wals zu sehen, der kaum größer war als sie selbst. Ich bemerkte, wie gesagt, ihre Hässlichkeit und seltsame Größe, doch dann beschloss ich, dass sie lediglich die erfundenen Götter eines primitiven Stammes von Fischern oder Seefahrern waren, eines Stammes, dessen letzte Nachkommen untergegangen waren, noch bevor der erste Vorfahre des Piltdown-Menschen oder des Neanderthalers geboren wurde. Ehrfürchtig vor diesem unerwarteten Einblick in eine Vergangenheit jenseits der Vorstellung der wagemutigsten Anthropologen stand ich grübelnd da, während der Mond sich seltsam in dem stillen Kanal vor mir spiegelte.

Dann sah ich es plötzlich. Mit nur einem leichten Brodeln, das seinen Aufstieg an die Oberfläche ankündigte, glitt das Ding in Sicht über den dunklen Wassern. Riesig, Polyphem gleich und abgrundtief hässlich, schoss es wie ein gewaltiges, Alpträumen entstammendes Monster zu dem Monolithen, um den es seine gigantischen, schuppigen Arme warf, während es seinen scheußlichen Kopf senkte und einige bedächtige Töne ausstieß. Ich glaube, da wurde ich wahnsinnig.

An meinen hektischen Aufstieg über den Hang und die Klippe und an meiner deliriösen Reise zurück zum gestrandeten Boot, erinnere ich mich nur flüchtig. Ich glaube, ich sang zu großen Teilen und lachte eigentümlich, wenn ich nicht fähig war, zu singen. Ich habe undeutliche Erinnerungen an einen großen Sturm, einige Zeit nachdem ich das Boot erreicht hatte, jedenfalls weiß ich, dass ich Donnerschläge und andere Geräusche gehört hatte, die die Natur nur in ihren wildesten Stimmungen von sich gibt.

Als ich aus den Schatten auftauchte, war ich in einem Hospital in San Francisco, hierher gebracht vom Kapitän eines amerikanischen Schiffes, das mein Boot mitten im Ozean aufgelesen hatte. In meinem Fieberwahn hatte ich viel geredet, doch fand ich, dass man meinen Worten nur geringe Aufmerksamkeit geschenkt hatte. Von einer Landerhebung im Paziik wussten meine Retter nichts, noch hielt ich es für nötig, auf etwas zu beharren, von dem ich wusste, dass sie es nicht glauben könnten. Ich suchte einmal einen berühmten Ethnologen auf und amüsierte ihn mit eigenartigen Fragen bezüglich der uralten, philistäischen Legende von Dagon, dem Fischgott, doch bald erkannte ich, dass er hoffnungslos konventionell war und drängteihn nicht weiter mit meinen Nachforschungen.

Es ist nachts, besonders wenn der Dreiviertelmond abnimmt, dass ich das Ding vor mir sehe. Ich habe Morpium ausprobiert, doch die Droge verschaffte mir nur kurzlebigen Nachlass und zog mich in ihre Umklammerung als einen hoffnungslosen Sklaven. Darum werde ich nun allem ein Ende setzen, nachdem ich diesen vollständigen Bericht zur Information oder zum verächtlichen Amüsement meiner Mitmenschen verfasst habe. Oft frage ich mich ob es nicht alles nur ein reines Hirngespinst gewesen sein kann --- eine bloße Fieberlaune während ich durch die Sonne geschwächt und verrückt in dem offenen Boot lag nach meiner Flucht von dem deutschen Kriegsschiff. Dies frage ich mich, doch erscheint mir immmer nur eine abscheulich lebhafte Vision als Antwort. Ich kann nicht an die Tiefsee denken ohne zu erzittern vor den namenlosen Dingen, die genau in diesem Moment auf ihrem schleimigen Grund kriechen und zappeln, ihre uralten Steingötzen anbeten und ihre eigenen widerwärtigen Abbilder in unterseeische Obelisken aus von Wasser durchdrungenem Granit schnitzen. Ich träume von einem Tag an dem sie sich über die Wellen erheben um in ihren stinkenden Fängen die Überbleibsel einer mickrigen, durch Krieg entkräfteten Menschheit herab zu ziehen --- von einem Tag an dem das Land versinken und der Meeresboden aufsteigen wird in Mitten von allumfassendem Pandämonium.

Das Ende ist nah. Ich höre ein Geräusch an der Tür, wie von einem gewaltigen, glitschigen Körper, der dagegen trampelt. Es wird mich nicht finden! Gott, \textit{diese Hand!} Das Fenster! Das Fenster!
\end{document}
