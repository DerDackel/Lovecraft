\documentclass[a4paper]{memoir}
\usepackage[ngerman]{babel}
\usepackage[T1]{fontenc}
\usepackage[utf8]{inputenc}
\usepackage{textcomp}

\begin{document}
\title{Die Katzen von Ulthar}
\author{H.P. Lovecraft\\
		Sebastian Jackel (Übersetzung)}
\date{}
\maketitle

\textit{Dieser Text steht unter einer Creative Commons CC-BY-SA Lizenz
(siehe\\ creativecommons.org/licenses/by-sa/3.0/). Das Englische Original von H.P. Lovecraft ist Teil der Public Domain. Die dieser Übersetzung zugrunde liegende Originalfassung kann unter hplovecraft.com nachgelesen werden.}

\vspace{12pt}

% ...Ulthar, jenseits des Flusses Skai... ?
Man sagt, dass in Ulthar, welches jenseits des Flusses Skai liegt, kein Mensch eine Katze töten darf. Und dies kann ich wahrlich glauben, wenn ich sie beobachte, wie sie schnurrend vor dem Feuer sitzt. Denn die Katze ist geheimnisvoll und sonderbaren Dingen nahe, die der Mensch nicht zu sehen vermag. Sie ist die Seele des antiken Aegyptus und Überbringer von Geschichten aus lang vergessenen Städten in Meroe und Ophir. Sie ist verwandt mit den Herren des Dschungels und Erbin der  Geheimnisse des uralten und finsteren Afrika. Die Sphinx ist ihre Base und sie spricht ihre Sprache, doch ist die Katze noch älter als die Sphinx und erinnert sich an das, was jene längst vergessen.

In Ulthar lebten, bevor die Bürger das Töten von Katzen verboten, ein alter Kötter und seine Frau, die sich daran erfreuten, die Katzen ihrer Nachbarn zu fangen und umzubringen. Warum sie dies taten, weiß ich nicht; außer dass Mancher die Stimme der Katze in der Nacht hasst und es übel aufnehmen, dass Katzen in der Dämmerung verstohlen durch Höfe und Gärten rennen. Doch was immer der Grund sein mochte, dieser alte Mann und seine Frau fanden Gefallen daran, jede Katze einzufangen und zu erschlagen, die ihrer Hütte zu nahe kam und manchen der Geräusche, die man des Nachts von dort vernehmen konnte, entnahmen viele Dorfbewohner, dass die Tötungsmethode außergewöhnlich absonderlich sei. Doch die Dorfbewohner besprachen diese Dinge nicht  mit dem alten Mann und seiner Frau, ob der gewöhnlichen Miene auf den verwelkten Gesichtern der zwei und, weil ihr Häuschen so klein und so finster versteckt unter breiten Eichenbäumen am Ende eines vernachlässigten Hofes. In Wahrheit hassten die Besitzer der Katzen diese merkwürdigen Leute, doch fürchteten sie sie noch mehr und anstatt sie brutale Meuchler zu schimpfen, gaben sie lediglich acht, dass kein geschätztes Haustier, oder ein Mäusejäger in Richtung der abgelegenen Hütte unter den dunklen Bäumen streunte. Wenn durch ein unvermeidliches Versehen eine Katze verloren gegangen und nach Einbruch der Dunkelheit Geräusche zu vernehmen waren, so klagte der Verlierer ohnmächtig, oder tröstete sich indem er dem Schicksal dankte, dass es nicht eines seiner Kinder war, die auf solche Weise verschwunden waren. Denn die Einwohner von Ulthar waren einfache Leute und wussten nicht, woher alle Katzen ursprünglich kamen.

Eines Tages betrat eine Karawane von fremdartigen Wanderern die gepflasterten Straßen von Ulthar. Dunkle Wanderer, die anders waren als alles andere fahrende Volk, welches zweimal im Jahr das Dorf passierte. Auf dem Marktplatz sagten sie für Silber die Zukunft voraus und kauften farbenfrohe Perlen von den Händlern. Aus welchem Land diese Wanderer ursprünglich kamen, wusste niemand, doch sah man sie fremdartigen Gebeten hingegeben und die Seiten ihrer Wagen trugen merkwürdige Zeichnungen von Figuren mit Menschenkörpern und den Köpfen von Katzen, Falken, Widdern und Löwen. Und der Anführer der Karawane trug einen Kopfschmuck aus zwei Hörnern mit einer eigentümlichen Scheibe dazwischen.

In dieser ungewöhnlichen Karawane gab es einen kleinen Jungen, der weder Vater noch Mutter hatte, sondern nur ein kleines schwarzes Kätzchen hegte. Die Pest hatte ihm übel mitgespielt und ihm doch dieses kleine, pelzige Ding gelassen um seinen Schmerz zu lindern; denn wer noch sehr jung ist,  kann großen Trost in den lebhaften Possen einer schwarzen Katze finden. So lächelte der Junge, den die dunklen Leute Menes nannten, öfter als dass er weinte während er mit seinem anmutigen Kätzchen auf den Stufen eines sonderbar bemalten Wagens spielte.

Am dritten Morgen, den die Wanderer in Ulthar verbrachten, konnte Menes sein Kätzchen nicht finden und wie er auf dem Marktplatz laut schluchzte, erzählten ihm einige Dorfbewohner von dem alten Mann und seiner Frau und von Geräuschen, die in der Nacht zuvor zu vernehmen gewesen waren. Und als er diese Worte hörte, wandelte sich sein Schluchzen zu Meditation und schließlich zum Gebet. Er streckte seine Arme zur Sonne aus und betete in einer Sprache, die kein Dorfbewohner verstand. Obwohl die Dorfbewohner auch nicht wirklich daran interessiert waren, ihn zu verstehen, da ihre Aufmerksamkeit vom Himmel und den merkwürdigen Formen, die die Wolken annahmen eingenommen wurde. Es war sehr eigenartig, doch als der kleine Junge seine Bitte geäußert hatte, schienen sich am Himmel schemenhafte, nebulöse, exotische Gestalten zu formen, von hybriden Kreaturen, gekrönt mit von Hörnern flankierten Scheiben.

In jener Nacht verließen die Wanderer Ulthar und wurden nie wieder gesehen. Und die Hauseigentümer waren besorgt, als sie bemerkten, dass im ganzen Dorf nicht eine einzige Katze mehr aufzufinden war. Von jedem Herd war die vertraute Katze verschwunden; Katzen groß und klein, schwarz und grau gestreift, gelb und weiß. Der Alte Kranon, der Bürgermeister schwor, dass das dunkle Volk die Katzen aus Rache für den Mord an Menes' Kätzchen mitgenommen hatte und verfluchte die Karawane und den kleinen Jungen. Doch Nith, der hagere Notar stellte fest, dass der alte Kötter und seine Frau am ehesten zu verdächtigen seien, denn ihr Hass auf Katzen war bekannt und wurde zunehmend wilder. Dennoch wagte es niemand, sich bei dem finsteren Paar zu beschweren. Auch nicht als der kleine Atal, Sohn des Schankwirts, schwor, in der Dämmerung alle Katzen von Ulthar im Hof der Alten unter den Bäumen gesehen hatte, wie sie sehr langsam und ernst in Zweierreihe im Kreis um die Hütte schritten, als würden sie dort ein unbekanntes, tierisches Ritual abhalten. Die Dorfbewohner wussten nicht, wie sehr sie einem kleinen Jungen glauben sollten und obwohl sie fürchteten, dass das böse Paar die Katzen durch Verzauberung in den Tod gelockt hatte, bevorzugten sie es, den alten Kötter nicht zu konfrontieren, bis sie ihn außerhalb seines dunklen, abweisenden Hofes trafen.

So ging man in Ulthar in nutzlosen Zorn vertieft zu Bett. Und als die Leute zum Morgengrauen wieder erwachten --- Siehe da! --- waren alle Katzen zurück an ihren angestammten Herd gekehrt! Groß und klein, schwarz und grau gestreift, gelb und weiß, keine einzige fehlte. Sehr geschmeidig und fett wirkten die Katzen und sie schnurrten klangvoll vor Zufriedenheit. Die Einwohner redeten miteinander über die Angelegenheit und staunten darüber nicht wenig. Der Alte Kranon bestand weiter darauf, dass das dunkle Volk die Katzen mitgenommen hatte, da Katzen von der Hütte des uralten Mannes und seiner Frau nicht lebend zurückkehrten. Doch alle gingen sich in einem einig: Die Weigerung aller Katzen, ihre Portion Fleisch zu essen oder ihr Schälchen Milch zu trinken war äußerst eigenartig. Und für zwei ganze Tage wollten die glatten, faulen Katzen von Ulthar kein Futter anrühren und nur am Feuer oder in der Sonne dösen.

Es dauerte eine ganze Woche bis die Dorfbewohner bemerkten, dass zur Abenddämmerung keine Lichter in den Fenstern der Hütte unter den Bäumen angingen. Dann bemerkte der hagere Nith, dass niemand den alten Mann oder seine Frau gesehen hatte, seit der Nacht in der die Katzen verschwunden waren. Nach einer weiteren Woche entschied der Bürgermeister, seine Furcht zu überkommen und in dienstlicher Sache die eigentümlich stille Behausung aufzusuchen, nahm jedoch aus Vorsicht Shang den Schmied und Thul den Steinmetz als Zeugen mit. Und als sie die morsche Tür aufgebrochen hatten, fanden sie nur dies vor: Zwei säuberlich abgenagte menschliche Skelette auf dem irdenen Boden und ein paar ungewöhnliche Käfer, die in den schattigen Ecken herumkrabbelten.

Später gab es großes Gerede unter den Bürgern von Ulthar. Zath der Leichenbeschauer, stritt lange mit Nith, dem hageren Notar und Kranon und Shang und Thul wurden mit Fragen überhäuft. Sogar der kleine Atal, der Sohn des Schankwirts, wurde genau befragt und bekam Zuckerwerk zur Belohnung. Man sprach über den alten Kötter und seine Frau, über die Karawane von dunklen Wanderern, vom kleinen Menes und seinem schwarzen Kätzchen, von Menes' Gebet und vom Zustand des Himmels während dieses Gebets, von den Taten der Katzen in der Nacht in der die Karawane aufgebrochen war und was man später in der Hütte unter den dunklen Bäumen in dem abstoßenden Hof fand.

Und am Ende beschlossen die Bürger jenes bemerkenswerte Gesetz, von dem Händler in Hatheg erzählen und über das Reisende in Nir sprechen; nämlich, dass in Ulthar kein Mensch eine Katze töten darf.

\end{document}
