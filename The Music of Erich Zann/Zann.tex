\documentclass[a4paper]{memoir}
\usepackage[ngerman]{babel}
\usepackage{textcomp}
\usepackage[cm-default]{fontspec}
\usepackage{xunicode}
\usepackage{xltxtra}
\setromanfont[Mapping=tex-text]{Linux Libertine O}

\begin{document}
\title{Die Musik des Erich Zann}
\author{H.P. Lovecraft\\
		Sebastian Jackel (Übersetzung)}
\date{}
\maketitle

\textit{Dieser Text steht unter einer Creative Commons CC-BY-SA Lizenz
(siehe Lizenztext unter creativecommons.org/licenses/by-sa/3.0/). Das Englische Original von H.P. Lovecraft ist Teil der Public Domain. Die dieser Übersetzung zugrunde liegende Originalfassung kann unter hplovecraft.com nachgelesen werden.}

\vspace{12pt}

Ich habe Karten der Stadt mit größter Sorgfalt studiert, doch habe ich die Rue d'Auseil nie wieder gefunden. Es handelte sich bei den Karten nicht nur um moderne --- ich weiß, dass Namen sich ändern können. Ich habe, im Gegenteil, alles Altertümliche zur Gegend tiefgreifend erforscht und persönlich jeden Ort jedweden Namens erkundet, der möglicherweise die Straße beherbergen könnte, die ich als Rue d'Auseil kenne. Doch trotz allem was ich getan habe, bleibt es eine schmachvolle Tatsache, dass ich weder das Haus noch die Straße oder auch nur den Ortsteil finden kann in dem ich während der letzten Monate meines ärmlichen Lebens als Student der Metaphysik an der Universität der Musik des Erich Zann gelauscht hatte.

Dass mein Gedächtnis löchrig ist, wundert mich nicht, denn meine Gesundheit, physisch wie geistig wurde erheblich geschwächt während der Zeit meines Aufenthalts in der Rue d'Auseil und ich erinnere mich, dass ich keine meiner wenigen Bekanntschaften je dorthin brachte. Doch dass ich den Ort nicht wieder finden kann ist zugleich bemerkenswert und verblüffend, denn er lag nur eine halbe Stunde Fußmarsch von der Universität entfernt und zeichnete sich durch Eigentümlichkeiten aus, die jemand, der einmal dort gewesen ist kaum vergessen konnte. Ich habe nie jemanden getroffen, der die Rue d'Auseil gesehen hat.

Die Rue d'Auseil lag jenseits eines dunklen Flusses, an dessen steiles Ufer sich geziegelte, trübverglaste Lagerhäuser säumten und der von einer massiven Brücke aus dunklem Stein überspannt wurde. Es war entlang dieses Flusses stets schummrig, als ob der Rauch der benachbarten Fabriken die Sonne fortwährend verdeckte. Der Fluss verpestete außerdem die Luft üblen Gerüchen, die ich niemals anderswo gerochen habe und die mir eines Tages helfen mögen, ihn zu finden, denn ich sollte sie auf der Stelle erkennen. Jenseits der Brücke lagen enge, gepflasterte Straßen mit Geländern und dann begann der Anstieg, zunächst langsam, doch wurde er unglaublich steil wenn man die Rue d'Auseil erreichte.

Ich habe nie eine andere Straße gesehen, so eng und steil wie die Rue d'Auseil. Es handelte sich beinahe um eine Klippe, allen Fahrzeugen verschlossen, an mehreren Stellen aus Treppen bestehend und am Gipfel in einer stolzen, efeubewachsenen Mauer endend. Das Pflaster war unregelmäßig und setzte sich teils aus Steinplatten, teils aus Kopfsteinpflaster und teils aus blanker Erde mit schwache,, graugrünem Bewuchs zusammen. Die Häuser waren hoch, mit spitzen Dächern, unglaublich alt und schienen absonderlich nach hinten, vorne oder seitwärts zu lehnen. Manchmal traf sich ein Paar gegenüberstehender Gebäude, die beide in die Gasse hinein lehnten fast wie ein Torbogen über der Straße und natürlich hielten sie das meiste Licht vom Boden darunter fern. Es gab auch einige Brücken von Haus zu Haus über die Straße.

Die Einwohner der Straße beeindruckten mich besonders. Zunächst dachte ich, es läge daran, dass sie alle still und verschwiegen waren, doch später entschied ich, dass es daran lag, dass sie alle sehr alt waren. Ich weiß nicht, wie es sich begab, dass ich in solch einer Straße lebte, doch war ich nicht ich selbst als ich hierher zog. Ich hatte schon an vielen armseligen Orten gewohnt und war aus Geldmangel stets von dort verwiesen worden bis ich schließlich bei dem wackeligen Haus in der Rue d'Auseil ankam, das vom gelähmten Blandot betreut wurde. Es war das dritte Haus vom oberen Ende der Straße aus und bei weitem das höchste von allen.

Mein Raum lag im vierten Stock und war das einzige Zimmer, das dort bewohnt wurde, da das Haus fast leer stand. In der Nacht meiner Ankunft hörte ich seltsame Musik aus der spitzen Dachstube über mir und am nächsten Tag fragte ich den alten Blandot danach. Er erzählte mir, dass es sich dabei um einen alten deutschen Cellisten handele, ein merkwürdiger, stummer Mann, der auf den Namen Erich Zann hörte und abends im Orchester eines minderwertigen Theaters spielte. Er fügte hinzu, dass Zanns Verlangen danach, zu spielen wenn er in der Nacht vom Theater zurückkehrte der Grund sei, warum er dieses hohe und einsame Dachzimmer ausgesucht hatte, dessen eines Giebelfenster der einzige Punkt der Straße war, von dem aus man über den Endwall auf den Abhang und das Panorama dahinter schauen konnte.

Danach hörte ich Zann jede Nacht und obgleich er mich wach hielt, suchte mich die Seltsamkeit seiner Musik heim. Selbst wenig Ahnung von der Kunst habend, war ich doch sicher, dass keine seiner Harmonien irgendwie vergleichbar  zu irgendeiner Musik, die mir bekannt war schien und ich schloss daraus, dass er ein Komponist von höchst untypischer Begabung sein musste. Je länger ich zuhörte, desto mehr wurde ich fasziniert, bis ich nach einer Woche beschloss, die Bekanntschaft des alten Mannes zu machen.

Eines Nachts als er von seiner Arbeit zurückkam, fing ich Zann auf dem Flur ab und sagte ihm, dass ich ihn gern kennenlernen und ihm beiwohnen würde, wenn er spiele. Er war eine kleine, dürre, gebeugte Person in schäbigen Kleidern, mit blauen Augen, einem grotesken, satyrhaften Gesicht und einem fast kahlen Kopf und schien nach meinen ersten Worten zugleich erzürnt und verängstigt. Meine offensichtliche Freundlichkeit jedoch, brach schließlich das Eis und er wies mich widerwillig an, ihm über die dunklen, knarrenden, wackligen Stufen zum Dachboden zu folgen. Sein Zimmer, eines von nur zwei unter dem steilen Giebeldach, war auf der Westseite zu jener hohen Mauer hin, die das obere Ende der Straße bildete. Es war sehr groß und erschien noch größer durch seine außerordentliche Blöße und Vernachlässigung. An Möbeln standen dort lediglich ein schmales, eisernes Bettgestell, ein schmutziger Waschtisch, ein kleiner Tisch, ein großes Bücherregal, ein eiserner Notenständer und drei altmodische Stühle. Notenblätter häuften sich unordentlich auf dem Boden. Die Wände bestanden aus blanken Brettern und waren wahrscheinlich nie verputzt worden, während die Fülle an Staub und Spinnweben den Ort eher verlassen als bewohnt erscheinen ließ. Erich Zanns Welt der Schönheit lag offenkundig in einem fernen Kosmos der Vorstellungskraft.

Mir einen Platz zuweisend schloss der stumme Mann die Tür, schob den großen, hölzernen Riegel vor und zündete eine Kerze an um die, die er mitgebracht hatte, zu ergänzen. Dann hob er sein Cello aus seiner mottenzerfressenen Abdeckung und setzte sich in den am wenigsten unbequemen Stuhl. Er nutzte seinen Notenständer nicht, doch bot mir keine Auswahl an und spielte aus dem Gedächtnis, mich so für über eine Stunde mit Formen, die ich nie zuvor gehört hatte verzaubernd. Formen die er selbst kreiert haben musste. Ihre exakte Natur zu beschreiben ist für den musikalisch Unerfahrenen unmöglich. Es handelte sich um eine Art Fuge mit wiederkehrenden, fesselnden Passagen doch mir fiel keine davon besonders auf, da jene seltsamen Noten fehlten, die ich zu anderen Gelegenheiten von meinem Zimmer aus wahrgenommen hatte.

Diese unvergesslichen Noten hatte ich im Gedächtnis behalten und sie oft gesummt und falsch nachgepfiffen, so dass als der Spieler seinen Bogen schließlich ablegte, ich ihn fragte ob er einige davon spielen möge. Als ich meine Frage begann, verlor das runzlige, satyrhafte Gesicht all die gelangweilte Gelassenheit, die es während dem Spiel besessen hatte und schien dieselbe eigentümliche Mischung von Zorn und Angst zu zeigen, die ich bemerkt hatte, als ich den alten Mann zuerst angesprochen hatte. Einen Moment lang war ich versucht, meine Überredungskunst zu gebrauchen, die Launen des Alters leichtfertig abtuend und versuchte gar, jene sonderbare Stimmung in meinem Gastgeber durch das Pfeifen einiger der Formen, denen ich die Nacht zuvor gelauscht  hatte, zu wecken. Doch ich verfolgte diesen Kurs nur für einen Moment, denn als der stumme Musiker die Pfiffe erkannte, wurde sein Gesicht plötzlich durch einen Ausdruck jenseits jedweder Deutung verzerrt und seine lange, kalte, knochige Hand streckte sich aus um meinen Mund zu verschließen und diese krude Imitation zu ersticken. Indem er dies tat, zeigte sich seine Verschrobenheit weiterhin dadurch, dass er einen bestürzten Blick zu dem einsamen, verhangenen Fenster warf, als sei er furchtsam ob eines Eindringlings --- Ein Blick, der doppelt absurd schien, da das Dachzimmer hoch und von allen benachbarten Dächern aus unzugänglich stand, als einziger Punkt der steilen Straße, so hatte es mir der Concierge erzählt, von dem aus man die Mauer am Gipfel übersehen könne.

Der Blick des alten Mannes brachte mir Blandots Bemerkung in Erinnerung und ich verspürte willkürlich den Wunsch, heraus auf das weite und verwirrende Panorama aus mondbeschienenen Dächern und Lichtern der Stadt jenseits der Hügelkuppe zu blicken, das von allen Bewohnern der Rue d'Auseil lediglich dieser griesgrämige Musiker sehen konnte. Ich bewegte mich in Richtung des Fensters und hätte fast die unscheinbaren Vorhänge zur Seite gezogen, als der stumme Zimmerherr, mit einer angstvollen Wut noch größer als zuvor, sich mir in den Weg stellte und mich diesmal mit seinem Kopf zur Tür wies während er sich bemühte, mich mit beiden Händen dorthin zu zerren. Nun gänzlich angewidert von meinem Gastgeber, befahl ich ihm, mich loszulassen und sagte ihm, dass ich sofort gehen würde. Seine Umklammerung löste sich und als er meine Abscheu und Ärgernis sah, schien sein eigener Zorn zu versiegen. Er festigte seinen Griff wieder, aber diesmal auf eine freundliche Art, drängte mich auf einen Stuhl und ging dann mit einem Ausdruck der Wehmut zum übersäten Tisch, wo er viele Worte mit einem Bleistift im bemühten Französisch eines Ausländers schrieb.

Die Notiz, die er mir schließlich übergab, war eine Bitte um Toleranz und Vergebung. Zann sagte, dass er alt, allein und befallen von fremdartigen Ängsten und Nervenleiden im Zusammenhang mit seiner Musik und anderen Dingen sei. Es habe ihn gefreut, dass ich seiner Musik gelauscht habe und er wünschte, dass ich wieder käme und seine Verschrobenheiten nicht beachte. Aber er könne niemand anderem seine merkwürdigen Harmonien vorspielen und könne es weder verkraften, sie von jemand anderem zu vernehmen, noch könne er ertragen, dass irgendjemand etwas in seinem Zimmer anfasse. Ihm war bis zu unserer Unterredung im Flur nicht bewusst gewesen, dass ich sein Spiel von meinem Zimmer aus wahrnehmen konnte und fragte nun, ob ich es mit Blandot arrangieren könnte, ein niedrigeres Zimmer zu nehmen, wo ich ihn nachts nicht hören könne. Er würde, so schrieb er, den Unterschied in der Miete bestreiten.

Als ich dasaß und sein abscheuliches Französisch entzifferte, empfand ich Nachsicht mit dem alten Mann. Er war ein Opfer, physischer und nervöser Leiden, wie ich und meine metaphysischen Studien hatten mir Güte beigebracht. Durch die Stille drang ein leises Geräusch vom Fenster her --- der Fensterladen muss im Nachtwind geklappert haben --- und aus irgendeinem Grund zuckte ich fast genauso heftig zusammen wie Erich Zann. Als ich zu Ende gelesen hatte, schüttelte ich meinem Gastgeber die Hand und verließ ihn als Freund. Am nächsten Tag gab mir Blandot ein teureres Zimmer im zweiten Stock, zwischen der Wohnung eines alternden Geldverleihers und der eines achtbaren Sattlers. Im dritten Stock wohnte niemand.

Es dauerte nicht lange, bis ich bemerkte, dass Zanns Begierde nach meiner Gesellschaft nicht so groß war, wie sie geschienen hatte, als er mich zum Umzug aus dem vierten Stock zu überzeugen versucht hatte. Er bat mich nicht, ihn aufzusuchen und wenn ich ihn besuchte, schien er unruhig und spielte lustlos. Dies trug sich stets nachts zu --- tagsüber schlief er und er ließ niemanden ein. Meine Zuneigung zu ihm wuchs nicht, obwohl das Dachzimmer und die merkwürdige Musik eine seltsame Faszination auf mich ausübten. Ich besaß ein eigentümliches Verlangen, aus diesem Fenster heraus zu sehen, über die Mauer und den unbesehenen Abhang hinunter auf die glitzernden Dächer und Spitzen die sich dort ausbreiten müssen. Einmal ging ich nach oben zum Dachboden während der Öffnungszeiten des Theaters, als Zann weg war, doch seine Tür war verschlossen.

Was mir jedoch gelang war, das nächtliche Spiel des stummen, alten Mannes zu überhören. Zunächst schlich ich auf Zehenspitzen herauf in meinen alten vierten Stock, später wurde ich mutig genug, die letzte, knarrende Treppe zur spitzen Dachkammer herauf zu klettern. Dort im engen Flur, vor der verschlossenen Tür mit dem zugestopften Schlüsselloch und hörte dabei oft Töne, die mich mit einer undefinierbaren Furcht erfüllten --- die Furcht vor vagem Erstaunen und grüblerischer Geheimnisse. Es war nicht so, dass die Töne hässlich geklungen hätten, denn das waren sie nicht, doch sie beinhalteten Vibrationen, die an nichts auf dieser Erde erinnerten und die zu bestimmten Intervallen eine symphonische Qualität annahmen, die ich kaum als nur von einem Spieler hervorgebracht wahrnehmen konnte. Erich Zann war fürwahr ein Genie von wilder Kraft. Während die Wochen vergingen, wurde sein Spiel immer wilder, während der alte Musiker eine zunehmend verstörtere und verstohlenere Haltung annahm, die mitleidserregend anzusehen war. Er verweigerte nun zu jeder Zeit, mich zu empfangen und mied mich, wann immer wir uns im Treppenhaus trafen.

Dann, eines Nachts während ich an der Tür lauschte, hörte ich das Kreischen des Cellos zu einem chaotischen Gewirr von Tönen anschwellen, ein Pandämonium das mich dazu geführt hätte, an meinem eigenen erschütterten Verstand zu zweifeln, wäre da nicht von jenseits des versperrten Portals ein kläglicher Beweis gekommen, dass der Schrecken real war --- jener furchtbare, undeutliche Schrei, den nur ein Stummer ausstoßen konnte und der nur in Momenten äußerster Furcht oder Schmerzen laut wurde. Ich klopfte mehrmals an die Tür, doch erhielt keine Antwort. Danach wartete ich in dem schwarzen Flur, zitternd vor Kälte und Furcht, bis ich den kraftlosen Versuch des armen Musikers vernahm, sich mit Hilfe eines Stuhls vom Boden zu erheben. Im Glauben, dass er gerade wieder nach einem Ohnmachtsanfall zu Bewusstsein gekommen war, wiederholte ich mein Klopfen und rief beschwichtigend meinen Namen. Ich hörte, wie Zann zum Fenster stolperte und Läden und Vorhang schloss und dann zur Tür wankte, die er zögernd öffnete um mich einzulassen. Dieses Mal war seine Freude über meine Anwesenheit echt, denn sein verzerrtes Gesicht leuchtete vor Erleichterung während er sich an meinen Mantel klammerte wie ein Kind sich an den Rockzipfel seiner Mutter klammert.

Erbärmlich zitternd zwang mich der alte Mann auf einen Stuhl, während er auf einen anderen sank, neben dem sein Cello und der Bogen achtlos auf dem Boden lagen. Er saß für einige Zeit still, merkwürdig nickend, doch mit der paradoxen Andeutung angestrengten und bangen Zuhörens. Später schien er zufrieden gestellt, durchquerte den Raum zu einem Stuhl beim Tisch, schrieb eine kurze Notiz, übergab sie mir und kehrte zum Tisch zurück, wo er begann, zügig und unaufhörlich zu schreiben. Die Notiz bat mich inständig bei aller Gnade und meiner eigenen Neugier halber hier zu warten, während er einen vollständigen Bericht in Deutsch über all die Wunder und Schrecken, die ihn heimsuchten verfasste. Ich wartete und des stummen Mannes Bleistift flog über das Papier.

Es war vielleicht eine Stunde später, während ich noch immer wartete und die fieberhaft beschriebenen Zettel des alten Musikers sich nach wie vor auftürmten, dass ich sah, wie Zann zusammenzuckte als wie durch die Spur einer fürchterlichen Erschütterung. Zweifellos schaute er zu dem verhangenen Fenster und lauschte schaudernd. Dann glaubte ich fast, selbst einen Ton zu vernehmen, obwohl es sich nicht um einen schrecklichen Ton handelte, sondern eine exquisite tiefe und unendlich entfernte Note, die einen Spieler in einem der benachbarten Häuser oder in einem Domizil jenseits der stolzen Mauer die ich noch nie hatte überblicken können suggerierte. Auf Zann wirkte ihr Einfluss entsetzlich, denn er ließ seinen Stift fallen und erhob er sich plötzlich, ergriff sein Cello und begann, die Stille der Nacht mit dem wildesten Spiel zu zerreißen, das ich je aus seinem Bogen vernommen hatte, außer beim Lauschen an der versperrten Tür.

Es wäre zwecklos, das wilde Spiel Erich Zanns in dieser entsetzlichen Nacht zu beschreiben. Es war fürchterlicher als alles, was ich je überhört hatte, da ich nun den Ausdruck in seinem Gesicht sehen und erkennen konnte, dass sein Antrieb in blanker Furcht lag. Er versuchte, Lärm zu machen um etwas zu bannen oder etwas zu übertönen --- was es war, konnte ich nicht ermessen, doch fühlte ich, dass es furchteinflößend sein musste. Das Spiel wurde fantastisch, wahnsinnig und hysterisch, doch behielt es bis zuletzt die Qualitäten höchsten Genies, die nach meinem Wissen diesem alten Manne innewohnten. Ich erkannte die Melodie --- es war ein wilder Ungarischer Tanz, beliebt in den Theatern und ich sann für einen Moment darüber, dass dies das erste Mal war, dass ich Zann je das Werk eines anderen Komponisten hatte spielen hören.

Lauter und Lauter, wilder und wilder schwoll das Schreien und Heulen des verzweifelten Cellos. Der Spieler troff vor unheimlichem Schweiß und verdrehte sich wie ein Affe, immerfort krampfhaft zu dem abgehängten Fenster starrend. In seinen fieberhaften Strapazen konnte ich beinahe schemenhafte Satyrn und Bacchanalien erkennen, tanzend und irrsinnig durch siedende Abgründe aus Wolken, Rauch und Blitzen wirbelnd wirbelnd. Und dann glaubte ich, eine schrillere, beständigere Note zu hören, die nicht von dem Cello kam, eine ruhige, bedachte, zielgerichtete, spöttische Note aus weiter Ferne im Westen.

In diesem Augenblick begannen die Fensterläden zu klappern im heulenden Nachtwind, der draußen wie in einer Antwort auf das wahnsinnige Spiel drinnen aufgekommen war. Zanns schreiendes Cello übertraf sich nun selbst und gab Töne von sich, von denen ich nie gedacht hätte, dass ein Cello sie von sich geben könne. Die Läden klapperten lauter, lösten sich und begannen gegen das Fenster zu schlagen. Dann brach das Glas klirrend unter den anhaltenden Schlägen und der kalte Wind drang herein, ließ die Kerzen flackern und die Blätter auf dem auf Tisch rasseln, wo Zann begonnen hatte, sein schreckliches Geheimnis niederzuschreiben. Ich schaute zu Zann und sah, dass er sich jenseits aller bewussten Wahrnehmung befand. Seine blauen Augen waren hervorgetreten, glasig und ohne Sehkraft und das hektische Spiel war zu einer blinden, mechanischen, unkenntlichen Orgie geworden, die keine Feder nur ansatzweise beschreiben könnte.

Ein plötzlicher Windstoß, stärker als die vorigen, wehte das Manuskript auf und zum Fenster hin. Ich folgte den umherfliegenden Blättern in Verzweiflung, doch waren sie davon bevor ich die zerschmetterten Scheiben erreichte. Da erinnerte ich mich meines alten Wunsches, aus diesem Fenster zu schauen, dem einzigen Fenster in der Rue d'Auseil von dem aus man den Abhang jenseits der Mauer und die sich darunter erstreckende Stadt betrachten konnte. Es war sehr dunkel, doch die Lichter der Stadt brannten immer und ich erwartete, sie dort zwischen Wind und Regen zu sehen. Doch als ich aus diesem höchsten aller Giebelfenster schaute, während die Kerzen flackerten und das Cello mit dem Nachtwind heulte, sah ich keine Stadt vor mir ausgebreitet und keine freundlichen Lichter aus mir bekannten Straßen leuchten sondern nur die Schwärze unermesslichen Raumes, unvorstellbaren Raumes, lebendig vor Bewegung und Musik und ohne jede Ähnlichkeit zu allem Irdischen. Und wie ich vor Entsetzen starrend dastand, blies der Wind die Kerzen in dem uralten, spitzen Dachzimmer aus und ließ mich in schonungsloser und undurchdringlicher Dunkelheit, mit Chaos und Pandämonium vor mir und dem dämonischen Wahnsinn des die Nacht anbellenden Cellos hinter mir.

Ich taumelte im Dunkel zurück, ohne die Möglichkeit, ein Licht anzufachen, stieß gegen den Tisch, warf einen Stuhl um und tastete mich schließlich zurück zu dem Ort, wo die Schwärze vor entsetzlicher Musik schrie. Mich selbst und Erich Zann zu retten konnte ich zumindest versuchen, trotz aller Kräfte, die mir entgegenstanden. Einmal glaubte ich, von einem kalten Ding gestreift zu werden und schrie, doch mein Schrei war unter dem abscheulichen Cello nicht zu hören. Plötzlich schlug mich aus der Dunkelheit der wild sägende Bogen und ich wusste, ich war seinem Spieler nahe. Ich tastete mich vor und fühlte die Rückseite von Zanns Stuhl und dann fand und schüttelte ich seine Schulter im Bemühen, ihn zu Verstand zu bringen.

Er reagierte nicht und immer noch kreischte das Cello ohne Unterlass. ich bewegte meine Hand zu seinem Kopf, dessen mechanisches Nicken ich anhalten konnte und rief in sein Ohr, dass wir beide vor diesen unbekannten Dingen der Nacht fliehen müssten. Doch weder antwortete er mir, noch verminderte er die Raserei seiner unbeschreiblichen Musik, während durch den ganzen Dachboden fremdartige Windströme in Dunkel und Chaos zu tanzen schiene. Als meine Hand sein Ohr berührte, erzitterte ich, doch ich konnte nicht sagen warum --- bis ich sein regungsloses Gesicht fühlte, das eiskalte, versteifte, atemlose Gesicht, dessen glasige Augen nutzlos in die Leere starrten.Und dann, wie durch ein Wunder fand ich die Tür und den großen, hölzernen Riegel und stürzte unbändig hinweg von dem gläsern blickenden Ding im Dunkeln und von dem schaurigen Geheul dieses verfluchten Cellos, dessen Heftigkeit sich noch steigerte, sogar als ich floh.

Springend, Gleitend, Fliegend die endlosen Treppen durch das dunkle Haus herab; ohne zu überlegen raus in die enge, steile und uralte Straße aus Stufen und wankenden Häusern rennend; die Stufen hinab und über Kopfstein klappernd zu den unteren Straßen und dem fauligen, von Häuserschluchten eingefassten Fluss; hechelnd über die große, dunkle Brücke zu den breiteren, heilsameren Straßen und bekannten Boulevards; all diese erschreckenden Eindrücke verweilen in mir. Und ich erinnere mich, dass es windstill war und das kein Mond am Himmel stand und dass alle Lichter der Stadt flimmerten.

Trotz meiner sorgfältigsten Suchen und Untersuchungen, war ich seither nicht im Stande die Rue d'Auseil zu finden. Aber es tut mir nicht gänzlich leid, weder darum, noch um den Verlust in traumlosen Abgründen, der eng beschriebenen Blätter, die alleine dazu in der Lage gewesen wären, eine Erklärung zu liefern für die Musik des Erich Zann.

\end{document}
